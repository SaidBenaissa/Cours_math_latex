\documentclass[landscape,11pt]{article}

\usepackage{pablo}
\usepackage[a4paper]{geometry}
\usepackage{savetrees}

\usepackage{tabularx}

\title{Progression secondes 2013 --- 2014\\\emph{\normalsize Version de travail}}
\author{}
\date{}

\begin{document}

\maketitle
\thispagestyle{empty}

\begin{tabularx}{\textwidth}{c|lXr}
1 &
Fonctions 1 & Intervalles et Généralités sur les fonctions : images antécédents, tableaux de variations, lecture graphiques. &
4 semaines\\\\\hline

2 &
Vecteurs 1 & Définition (translation + vecteur), Chasles, somme de deux vecteurs &
2 semaines
\\\\\hline

3 &
Statistiques 1 & Diagrammes, moyenne, médiane, quartiles &
2 semaines
\\\\\hline

4 &
Fonctions 2 & Affines, linéaires, résolution d’équations et inéquations du premier degré, algébriquement et graphiquement, tableau de signe d’un produit de facteurs du premier degré &
3 semaines	
\\\hline

5 &
Vecteurs 2 & Analytique (repérage, coord. milieu, distance 2 points, coord. vecteurs), multiplication par un réel, vecteurs colinéaires &
3 semaines
\\\hline

6 &
\multicolumn{2}{l}{Systèmes linéaires, équation de droite} &
2 semaines
\\\\\hline

7 &
Fonction 3 & Fonction carré et polynôme du second degré, résolution équations et inéquations (alg. et graph.) correspondantes + graphique &
4 semaines
\\\hline

8 &
Espace & &
3 semaines
\\\\\hline

9 &
Probabilités & &
2 semaines
\\\\\hline

10 &
Fonctions 4 & Fonction inverse, homographique résolution équations et inéquations (valeurs interdites, tableau de signe) + graphique &
2 semaines
\\\hline

11 &
Statistique 2 & Échantillonnage &
1,5 semaine
\\\\\hline

12 &
Trigonométrie &&
1 semaine
\\\\

\end{tabularx}
\end{document}
