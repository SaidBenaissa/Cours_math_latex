\documentclass[12pt]{article}

\usepackage{pablo}
\usepackage[landscape,a5paper,margin=1cm]{geometry}
\pagestyle{empty}
\usepackage{tabularx}

%\renewcommand{\tabularxcolumn}[1]{>m{#1}}
\def\tabularxcolumn#1{m{#1}}
\newcolumntype{C}{>{\centering\arraybackslash}X}

\newcommand{\droite}{\begin{tikzpicture}[very thick,baseline={($(current bounding box.center)+(0,-2.5ex)$)}]
  \draw[-latex] (-3,0) -- (3,0);
  \foreach \i in {-2.5, -2, ..., 2.5} {
    \draw (\i,.1) -- (\i,-.1);
  }
  \draw (-1,0) node[above]{\small $a$};
  \draw (1.5,0) node[above]{\small $b$};
\end{tikzpicture}}

\begin{document}
$a$ et $b$ sont deux nombres, tels que $a<b$.

\begin{tabularx}{\textwidth}{|c|C|C|}
  \hline
  Inéquation(s) & Intervalles (et simplification) & Droite des réels\\
  \hline
  \hline
  $x<a$ & & \droite \\
  \hline
  $x\leq a$ & & \droite \\
  \hline
  $x>a$ & & \droite \\
  \hline
  $x\geq a$ & & \droite \\
  \hline
  \hline
  $x<a$ et $x>b$ & & \droite \\
  \hline
  $x<a$ ou $x>b$ & & \droite \\
  \hline
  $x>a$ et $x<b$ & & \droite \\
  \hline
  $x>a$ ou $x<b$ & & \droite \\
  \hline
  $x>a$ et $x>b$ & & \droite \\
  \hline
  $x>a$ ou $x>b$ & & \droite \\
  \hline
  $x<a$ et $x<b$ & & \droite \\
  \hline
  $x<a$ ou $x<b$ & & \droite \\
  \hline
\end{tabularx}

\end{document}
