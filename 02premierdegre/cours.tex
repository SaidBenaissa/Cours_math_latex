\section{Équations du premier degré}

TODO

\begin{definition}
  Soit une équation d'inconnue $x$. Résoudre cette équation consiste à trouver
  toutes les valeurs de $x$ (appelées \emph{solutions}) qui vérifient
  l'équation.
\end{definition}

\subsection{Équations du premier degré}

\begin{definition}
  Une équation du premier degré est une équation de la forme $ax+b=0$.
\end{definition}

\begin{remarque}
  Résoudre une équation du premier degré $ax+b=0$ revient à trouver les abscisses des points d'intersection de la fonction $f:x\mapsto ax+b$ avec l'axe des abscisses.
\end{remarque}

\begin{propriete}[Résolution algébrique]
  Soit une équation $ax+b=0$.

  \begin{itemize}
    \item Si $a\neq0$, l'unique solution est $x=-\frac{b}{a}$.
    \item Si $a=0$ et $b\neq0$, l'équation n'a pas de solutions.
    \item Si $a=0$ et $b=0$, tous les réels sont solutions.
  \end{itemize}
\end{propriete}

\begin{methode}[Résolution graphique]
  Soit une équation $ax+b=0$. Pour résoudre cette équation graphiquement, on trace la courbe représentative $\cal D$ de la fonction affine $f:x\mapsto ax+b$.
  Si elles existent, les solutions de l'équation sont les abscisses des points d'intersection de $\cal D$ avec l'axe des abcsisses.
  Trois cas sont possibles.
  \begin{itemize}
    \item Si la droite $\cal D$ n'est pas parallèle à l'axe des abscisses, il y a une unique solution.
    \item Si la droite $\cal D$ est parallèle à l'axe des abscisses, et distincte de celui-ci, l'équation n'a pas de solutions.
    \item Si la droite $\cal D$ est confondue avec l'axe des abscisses, l'équation a une infinité de solutions : l'ensemble des réels.
  \end{itemize}
\end{methode}


\section{Inéquations du premier degré}

\begin{propriete}[Signe d'une fonction affine]
  Soit une fonction affine $f:x\mapsto ax+b$.
  \begin{itemize}
    \item Si $a>0$, alors $f(x)$ est négatif si $x<-\frac{b}{a}$, et positif si $x>-\frac{b}{a}$.
    \item Si $a<0$, alors $f(x)$ est positif si $x<-\frac{b}{a}$, et négatif si $x>-\frac{b}{a}$.
    \item Si $a=0$, alors, pour tout $x\in{\mathbb R}$, $f(x)$ est du signe de $b$.
  \end{itemize}
\end{propriete}

\begin{methode}[Résolution graphique]
  Soit une inéquation $ax+b>0$, et la fonction affine $f:x\mapsto ax+b$. Les
  solutions de l'inéquation sont les abscisses des points de la courbe de $f$
  situés au dessus de l'axe des abscisses.
\end{methode}


\section{Intervalles de $\mathbb{R}$}

TODO

\section{Équations produit}

TODO

\begin{propriete}[Équation produit]
  Soient $A$ et $B$ deux réels. Alors $A\times B=0$ si et seulement si $A=0$ ou $B=0$.

  En particulier, $(ax+b)(cx+d)=0$ si et seulement si $ax+b=0$ ou $cx+d=0$.
\end{propriete}
