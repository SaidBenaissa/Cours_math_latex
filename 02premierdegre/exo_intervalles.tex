\documentclass[14pt, aspectratio=169]{beamer}

\usepackage[utf8]{inputenc}
\usepackage[T1]{fontenc}
\usepackage[francais]{babel}
\usepackage[]{tikz}
\usepackage[]{multicol}
\usetikzlibrary{calc}

\useoutertheme{infolines}
\usecolortheme{crane}

\institute{Lycée Marie Curie}
\date{2014 --- 2015}
\logo{}
\title{(In)Équations du premier degré}

\begin{document}


\section{Intervalles}
\begin{frame}
  Dans chacun des cas, représenter les solutions des couples d'inéquations sur la droite des réels, puis sous fome d'intervalles.
  \begin{enumerate}
    \item $x\geq 3$ et $x+2\leq 9$
    \item $3x\geq1$ et $8-x>$
    \item $x+1<2x+2$
    \item $2x>2$ et $x+1\geq 5$
    \item $3x-3<0$ et $2x+2\leq 3x-3$
  \end{enumerate}
\end{frame}

\begin{frame}
  Dans chacun des cas, représenter les solutions des couples d'inéquations sur la droite des réels, puis sous fome d'intervalles.
  \begin{enumerate}
    \item $2x>2$ ou $x+1\geq 5$
    \item $3x-3<0$ ou $2x+2\leq 3x-3$
    \item $x\geq 1$ ou $\frac{x}{4}<1$
  \end{enumerate}
\end{frame}

\begin{frame}
  \begin{multicols}{2}

  Un potager $ABCD$ rectangulaire est délimité par une cloture. On souhaite augmenter d'une même longueur $x$ chacune des dimensions du rectangle. Sachant que l'on ne dispose, en tout, que de 100~m de cloture, à quel intervalle doit appartenir $x$ ?

  \columnbreak

  \begin{center}
    \begin{tikzpicture}[scale=3,thick]
      \coordinate (A) at (0,0);
      \coordinate (B) at (1,0);
      \coordinate (C) at (1,.6);
      \coordinate (D) at (0,.6);
      \draw (A) node[below left]{$A$}
         -- (B) node[below]{$B$}
         -- (C) node[above right]{$C$}
         -- (D) node[left]{$D$}
         -- cycle;
      \coordinate (E) at ($(B)+(.5,0)$);
      \coordinate (F) at ($(E)+(0,{.5+.6})$);
      \coordinate (G) at ($(F)-(1.5,0)$);
      \draw[dashed] (B)
         -- (E) node[below right]{$E$}
         -- (F) node[above right]{$F$}
         -- (G) node[above left]{$G$}
         -- (D);
         \draw ($.5*(D)+.5*(G)$) node[left]{$x$};
         \draw ($.5*(B)+.5*(E)$) node[below]{$x$};
         \draw ($.5*(A)+.5*(D)$) node[left]{$6$};
         \draw ($.5*(A)+.5*(B)$) node[below]{$10$};
    \end{tikzpicture}
  \end{center}
\end{multicols}

\emph{Toutes les longueurs sont données en mètres.}
\end{frame}

\end{document}
