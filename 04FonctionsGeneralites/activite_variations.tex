\documentclass[12pt, aspectratio=43]{beamer}

\usepackage[utf8x]{inputenc}
\usepackage[T1]{fontenc}
\usepackage[francais]{babel}
\usepackage{tikz}

\useoutertheme{infolines}
\usecolortheme{crane}

\institute{Lycée Marie Curie}
\date{2014}
\logo{}
\title{Généralités sur les fonctions}

\begin{document}


\section{Variations}
\begin{frame}
  Ce graphique représente l'altitude atteinte par des marcheurs en fonction de la distance parcourue, pendant une randonnée de 5~km.

  \begin{enumerate}
    \item Sur quelles parties du parcours les marcheurs montent-ils ? Sur quelles parties descendent-ils ?
    \item Quelle est la plus haute altitude atteinte ? La plus basse ?
    \item Quelle est l'altitude maximale atteinte pendant les deux premiers kilomètres ?
    \item Dresser le \emph{tableau de variation} de l'altitude.
  \end{enumerate}

  \begin{tikzpicture}[ultra thick,xscale=2.2, yscale=.018]
    \draw[dotted, gray, xstep=.5, ystep=50] (0, 200) grid (5, 400);
    \draw[-latex] (0,200) -- (5.2,200);
    \draw[-latex] (0,200) -- (0,420);
    \foreach \x in {0, 1, ..., 5} {
      \draw (\x,200) node[below]{\x};
      \draw (\x,{200-10}) -- (\x,{200+10});
    }
    \foreach \y in {200, 300, ..., 400} {
      \draw (0,\y) node[left]{\y};
      \draw (-.1, \y) -- (.2, \y);
    }
    \draw [cyan] plot [smooth, tension=0.5] coordinates {
      (0,250)
      (.5, 270)
      (1, 300)
      (2, 280)
      (3, 400)
      (3.5, 350)
      (4, 380)
      (4.5, 280)
      (5, 250)
    };
    \draw (.6,375)  node[fill=white, opacity=.8, text opacity=1]{Altitude (m)};
    \draw (4.5,225) node[fill=white, opacity=.8, text opacity=1]{Distance (km)};
  \end{tikzpicture}

\end{frame}

\end{document}
