\documentclass[12pt]{article}

\usepackage{pablo}
\usepackage[a5paper,margin=1cm]{geometry}

\pagestyle{empty}

\begin{document}

\begin{center}
  \Large Utilisation de la calculatrice
\end{center}

Les objectifs de cette séance sont, en utilisant la calculatrice :
\begin{itemize}
  \item Savoir tracer la courbe d'une fonction, en choisissant l'intervalle.
  \item Savoir donner le tableau de valeurs d'une fonction.
  \item Savoir résoudre (de manière approchée) les problèmes suivants :
    \begin{itemize}
      \item Dresser le tableau de variations d'une fonction.
      \item Déterminer les valeurs maximum et minimum prises par une fonction.
      \item Résoudre des équations.
    \end{itemize}
\end{itemize}

Dans tout le problème, on considère les fonctions (définies sur $\mathbb{R}$) :
\[f:x\mapsto \sqrt{x^2}\]
\[g:x\mapsto \frac{x^3}{3}-14x^2+189x-810\]

\begin{exercice}[Tracé de courbes]~
  \begin{enumerate}
    \item Tracer l'allure de la courbe de la fonction $f$, sur l'intervalle $\left[ -3;3 \right]$.
    \item Tracer l'allure de la courbe de la fonction $g$ sur l'intervalle $\left[ 8; 20 \right]$. Tracer son tableau de variations.
  \end{enumerate}
\end{exercice}

\begin{exercice}[Tableau de valeurs]~
  \begin{enumerate}
    \item Donner le tableau de valeurs de la fonction $f$ de -4 à 4, avec un pas de 1.
    \item Donner le tableau de valeurs de la fonction $g$ de 14 à 18, avec un pas de 0,5.
  \end{enumerate}
\end{exercice}

\begin{exercice}
  \emph{Les réponses attendues sont des valeurs approchées.}
  \begin{enumerate}
    \item Quelles sont les solutions de $x^2+x-3=0$ ?
    \item Quelles sont les coordonnées des points d'intersection des fonctions $f$ et $g$ ?
  \end{enumerate}
\end{exercice}


\end{document}
