\documentclass[14pt, aspectratio=43]{beamer}

\usepackage[utf8]{inputenc}
\usepackage[T1]{fontenc}
\usepackage[francais]{babel}
\usepackage{tkz-tab}

\useoutertheme{infolines}
\usecolortheme{crane}

\institute{Lycée Marie Curie}
\date{2014}
\logo{}
\title{Généralités sur les fonctions}

\begin{document}


\section{Contrôle}
\begin{frame}
\end{frame}

\begin{frame}
  \begin{enumerate}
    \item Soit $f:x\mapsto 3x+2$.
      \begin{enumerate}
        \item Montrer que si $a<b$, alors $3a+2<3b+2$.
        \item En déduire que $f$ est croissante.
      \end{enumerate}
    \item Soit $g:x\mapsto \left( 3x+2 \right)\left( x-1 \right)$. Résoudre $g(x)=0$.
    \item Soit $h:x\mapsto \left( x+2 \right)^2$. Montrer que pour tout $x$ de $\mathbb{R}$,  $h(x)\geq0$. En déduire la valeur du minimum de $h$.
    \item Quels sont les extremums de la fonction suivante ?
  \begin{center}
    \begin{tikzpicture}[xscale=1,yscale=1]
      \tkzTabInit[lgt=1,espcl=2]
      {$x$ /1,
        $f$ /2
      }
      {-2, {-1,25}, 1, {1,5}, 3}%
      \tkzTabVar{+/2, -/0, +/1, -/-1, +/3}
    \end{tikzpicture}
  \end{center}
  \end{enumerate}
\end{frame}

\end{document}
