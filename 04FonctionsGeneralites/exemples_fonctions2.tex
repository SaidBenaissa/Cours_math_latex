
\documentclass[12pt, aspectratio=169]{beamer}

\usepackage[utf8x]{inputenc}
\usepackage[T1]{fontenc}
\usepackage[francais]{babel}
\usepackage{tikz}

\useoutertheme{infolines}
\usecolortheme{crane}

\institute{Lycée Marie Curie}
\date{2014}
\logo{}
\title{Généralités sur les fonctions}

\begin{document}


\section{Exemples}
\begin{frame}

  \frametitle{Exemple 3 --- Expressions algébrique}

  Soit $f$ la fonction définie sur $\mathbb{R}$ par $f:x\mapsto 3x-1$.
  \begin{itemize}
    \item L'image de -5 par $f$ est : \ldots
    \item Le(s) antécédent(s) de 3 par $f$ sont : \ldots
  \end{itemize}
\end{frame}

\begin{frame}

  \frametitle{Exemple 4 --- Tableau de valeurs}

  Soit $g$ la fonction définie sur $\mathbb{R}$, dont on connait les valeurs suivantes.

  \begin{center}
    \begin{tabular}{p{1cm}|c|c|c|c|c}
      $x$ & -1 & 0 & 1 & 5 & 10 \\
      \hline
      $g(x)$ & 1 & -2 & 0 & 4 & -2\\
    \end{tabular}
  \end{center}
  \begin{itemize}
  \item $g(0)=\cdots$
  \item $g(2)=\cdots$
    \item Le(s) antécédent(s) de -2 par $g$ sont : \ldots
  \end{itemize}
\end{frame}

\begin{frame}
  \frametitle{Exemple 5 --- Algorithme}

  Soit $h$ la fonction définie par l'algorithme suivant :

  \begin{quote}
    On prend un nombre.\\
    On ajoute 3 à ce nombre. \\
    On le divise par 2. \\
    On l'élève au carré. \\
    On affiche ce nombre.
  \end{quote}

  \begin{itemize}
    \item L'image de 5 par $h$ est \ldots
    \item L'image de 4 par $h$ est \ldots
  \end{itemize}
\end{frame}

\end{document}
