\documentclass{article}

\usepackage{pablo}
\usepackage{pablo-listings}
\usepackage[a5paper,margin=1.8cm]{geometry}
\usepackage{tabularx}

\pagestyle{empty}

\begin{document}

\begin{center}
    {\LARGE Dichotomie\\}

    \Large Approximation de la solution d'une équation
\hrule
\end{center}

L'objet de cet exercice est de trouver une solution approchée de l'équation $x^3-x^2-7x-8=0$.

\begin{enumerate}
    \item \emph{Approche graphique (5 minutes)}
        \begin{enumerate}
            \item À la calculatrice ou avec le logiciel \emph{Géogébra}, tracer la courbe représentative de la fonction $f:x\mapsto x^3-x^2-7x-8$. Tracer l'allure de la courbe sur votre compte-rendu.
            \item Combien de solutions semble avoir l'équation étudiée ?
            \item \label{encadrement} Donner un encadrement d'une des solutions.
        \end{enumerate}
    \item \emph{Algorithme de dichotomie à la main (20 minutes)}\label{main} L'algorithme de dichotomie permet de trouver une solution approchée de l'équation considérée (les points d'interrogation sont à remplacer par les valeurs trouvées à la question \ref{encadrement}).

        \begin{lstlisting}[language=naturel,frame=lines,mathescape=true]
        a <- ???
        b <- ???

        Pour n allant de 1 $à$ 10:
        Faire
            m <- $\frac{a+b}{2}$
            Si $f(a)$ et $f(m)$ $\text{sont de même signe}$
            Alors
                a <- m
            Sinon
                b <- m
            FinSi
        FinPour
        Afficher a, b
        \end{lstlisting}
        \begin{enumerate}
            \item Exécuter l'algorithme en complétant le tableau suivant, pour les trois premières itérations.

                \begin{tabularx}{\linewidth}{|X|X|X|X|X|X|X|}
                    \hline
                    n & m & a & b \\
                    \hline
                    \hline
                    XXXXX & XXXXX &   &   \\
                    \hline
                    1 &   &   &   \\
                    \hline
                    2 &   &   &   \\
                    \hline
                    3 &   &   &   \\
                    \hline
                \end{tabularx}
            \item Bilan : Donner un encadrement de la solution de l'équation.
        \end{enumerate}
    \item \emph{Mise en œuvre informatique de l'algorithme. (20 minutes)} Afin d'automatiser ce calcul, et de pouvoir obtenir une approximation bien plus précise, nous allons réaliser un programme informatique pour réaliser cette tâche.
        \begin{enumerate}
            \item Recopier le code Python suivant dans AmiensPython.
                \begin{lstlisting}[language=python,frame=lines,mathescape=true]
                def f(x):
                    return x**3-x**2+7*x-8

                print f(0)
                print f(2)
                \end{lstlisting}
                À quoi servent ces lignes ?
            \item Effacer les deux dernières lignes de votre programme (celles commençant par \texttt{print}), et recopier et compléter le code suivant, qui est une traduction en Python de l'algorithme de la question \ref{main} (remplacer les points d'interrogation par les bonnes valeurs).
                \begin{lstlisting}[language=python,frame=lines,mathescape=true]
                a = ???
                b = ???
                for i in range(10):
                    m = ???
                    if f(m) * f(a) > 0:
                        a = m
                    else:
                        b = m
                print (a, b)
                \end{lstlisting}
            \item Faire fonctionner cet algorithme, et donner un encadrement de la solution de l'équation.
            \item Modifier l'algorithme pour augmenter la précision de l'approximation, et donner un nouvel encadrement.
        \end{enumerate}
    \item \emph{Pour aller plus loin.}
        \begin{enumerate}
            \item Modifier l'algorithme pour qu'il s'arrête lorsque la solution proposée est correcte au dixième près. Au millième près.
            \item Expliquer en quoi le test $f(m)\times f(a)>0$ permet de tester si $f(m)$ et $f(a)$ sont de même signe.
            \item Modifier l'algorithme pour qu'il trouve une solution de $f(x)=5$ plutôt que de $f(x)=0$.
        \end{enumerate}
\end{enumerate}
\end{document}
