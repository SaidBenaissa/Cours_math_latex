\documentclass[12pt, aspectratio=43]{beamer}

\usepackage[utf8x]{inputenc}
\usepackage[T1]{fontenc}
\usepackage[francais]{babel}
\usepackage{tikz}

\useoutertheme{infolines}
\usecolortheme{crane}

\institute{Lycée Marie Curie}
\date{2014}
\logo{}
\title{Généralités sur les fonctions}

\begin{document}


\section{Variations}
\begin{frame}
  \begin{enumerate}
    \item Résoudre $x^2+x-3=x+1$.
    \item On a tracé sur le graphique suivant les courbes des fonctions $f:x\mapsto x^2+x-3$ et $g:x\mapsto x+1$. Lire graphiquement $f(-2)$, $g(-2)$, $f(2)$, $g(2)$.
    \item En déduire une méthode pour résoudre graphiquement une équation du type $f(x)=g(x)$.
  \end{enumerate}

  \begin{center}
  \begin{tikzpicture}[ultra thick,xscale=1.9, yscale=0.6]
    \draw[dotted,gray,xstep=1,ystep=1] (-3,-4) grid (3,4);
    \draw[-latex] (-3,0) -- (3,0);
    \draw[-latex] (0,-4) -- (0,4);
    \draw[scale=1,domain=-3:2.2,smooth,variable=\x,blue] plot ({\x},{\x*\x+\x-3});
    \draw[scale=1,domain=-3:3,smooth,variable=\x,brown] plot ({\x},{\x+1});
    \draw[blue] (1, -1) node[below right]{$\mathcal{C}_f$};
    \draw[brown] (1, 2) node[above left]{$\mathcal{C}_g$};

    \foreach \x in {-3, -2, -1, 1, 2, 3} {
      \draw (\x, 0) node[below]{\x};
    }
    \foreach \y in {-3, -1, 1, 3} {
      \draw (0, \y) node[left]{\y};
    }
    \draw (0,0) node[below left]{$O$};

  \end{tikzpicture}
\end{center}

\end{frame}

\end{document}
