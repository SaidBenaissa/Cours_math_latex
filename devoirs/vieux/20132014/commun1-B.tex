\documentclass[12pt]{article}

\usepackage[utf8]{inputenc}
\usepackage[T1]{fontenc}
\usepackage[frenchb]{babel}

\usepackage{amsmath}
\usepackage{amsthm}
\theoremstyle{definition}
\newtheorem{exercice}{Exercice}

\usepackage{tikz}
\usepackage{multicol}

\pagestyle{empty}

\begin{document}

\setcounter{exercice}{3}
\begin{exercice}~

\begin{enumerate}
  \item R\'esoudre les in\'equations suivantes, et donner les solutions \`a l'aide d'intervalles.
    \begin{multicols}{2}
      \begin{enumerate}
        \item $4>x+5$
        \item $2x+10\leq6$
        \item $6(x+1)\geq x-3$
        \item $6(x+1)\geq x-3$ et $4>x+5$
        \item $6(x+1)\geq x-3$ ou $2x+10\leq6$
        \item $4>x+5$ ou $2x+10\leq6$
      \end{enumerate}
    \end{multicols}

  \item R\'epondre par vrai ou faux apr\`es avoir r\'esolu l'\'equation.
    \begin{enumerate}
      \item La solution de $\dfrac{2x+1}{5}=\dfrac{2x-1}{7}$ est un nombre entier naturel.
      \item La solution de $2x-\dfrac{5-x}{2}=2$ est un nombre entier naturel.
    \end{enumerate}

  \item Est-il possible de construire la figure suivante, o\`u :
      $ABCD$ et $AEFG$ sont des carr\'es ;
      $AE=x$ ;
      $CD=3x+4$ ;
      $BE = 3$ ?
    Justifiez votre r\'eponse.

    \begin{center}
      \begin{tikzpicture}
        \draw (0,0) node[below left]{$A$}
          -- (0,3) node[above left]{$B$}
          -- (3,3) node[above right]{$C$}
          -- (3,0) node[below right]{$D$}
          -- cycle;

        \draw (0,1) node[left]{$E$}
          -- (1,1) node[above right]{$F$}
          -- (1,0) node[below]{$G$};
      \end{tikzpicture}
    \end{center}

  \item Pour organiser le devoir commun de secondes, les professeurs de
    math\'ematiques doivent r\'epartir les \'el\`eves dans des salles de classe
    identiques. S'ils utilisent huit salles, il y aura sept places libres dans
    chaque salle. S'ils utilisaient quatre salles, il manquerait 72 places.

    On appelle $s$ le nombre de places des salles, et $n$ le nombre total d'\'el\`eves.

    Quelle est le nombre de places par salle ?

\end{enumerate}
\end{exercice}

\end{document}
