\documentclass[11pt]{article}

\usepackage{pablo}
\usepackage{yhmath}
  \usetikzlibrary{3d,calc}

\usepackage[a5paper,margin=1.4cm]{geometry}

\pagestyle{empty}
\begin{document}

\begin{center}
  \textsc{DM}
  ---
  {
    \Large
    Probabilités

    ~
    \hrule
  }
\end{center}

\begin{em}
  À rendre le vendredi 18 avril.
\end{em}

\begin{exercice}
  J'ai écrit un programme sur ma calculatrice, qui affiche aléatoirement un des nombres 0, 1, 2 ou 3. On connait les probabilités suivantes :
  \begin{itemize}
    \item le nombre affiché est pair : $^5/_6$ ;
    \item le nombre affiché est strictement positif : $^3/_4$ ;
    \item le nombre affiché est 3 : $^1/_6$.
  \end{itemize}
  Quelle est la probabilité d'obtenir 1 ?
\end{exercice}

\begin{exercice}
  Une urne contient quatre boules indiscernables au toucher, numérotées de 1 à 4. On tire successivement deux boules, sans remise.
  \begin{enumerate}
    \item Donner la taille de l'univers de cette expérience aléatoire.
    \item Quelle est la probabilité que la seconde boule porte un numéro plus grand que la première ?
    \item Quelle est la probabilité que la somme des deux boules fasse 4 ?
  \end{enumerate}
\end{exercice}

\begin{exercice}~
  \begin{enumerate}
    \item On lance une pièce de monnaie deux fois de suite. Cette pièce est mal équilibrée : il y a une chance sur trois de faire pile, et deux chances sur trois de faire face.
      \begin{enumerate}
        \item Combien y a-t-il d'issues à cette expérience aléatoire ?
        \item Soit $A$ l'évènement « obtenir deux fois pile ». Déterminer $P(A)$.
        \item Soit $B$ l'évènement « obtenir exactement une fois face ». Déterminer $P(B)$.
        \item Décrire $\bar B$. Calculer sa probabilité.
      \end{enumerate}
    \item On lance deux pièces de monnaie indistinguables, donnant pile et face avec les mêmes probabilités que la pièce de la question précédente. Répondre aux même questions que pour la première expérience.
  \end{enumerate}
\end{exercice}

\end{document}
