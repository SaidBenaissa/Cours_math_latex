\documentclass[11pt]{article}

\usepackage{pablo}
\usepackage{pablo-listings}
\usepackage[a5paper,margin=1.3cm]{geometry}
\usepackage{enumerate}
\usepackage{multicol}
\usepackage{framed}

\pagestyle{empty}
\begin{document}

  \section*{Devoir à la maison --- Correction}
  \emph{Rendu le vendredi 8 novembre}
  \setcounter{exercice}{0}

  \begin{exercice}[Vecteurs] Exercice 90 page 178.


    \begin{enumerate}[a.]
      \item \emph{Montrer que $\vecteur{BI}=\vecteur{KD}$.}

        Cette égalité fait intervenir le point $I$, à propos duquel la seule
        information que nous ayons est que c'est le symétrique de $A$ par
        rapport à $B$.  Donc $B$ est le milieu du segment $[AI]$, et
        $\vecteur{AB}=\vecteur{BI}$.

        Pour les mêmes raisons, $\vecteur{KD}=\vecteur{DC}$.

        Enfin, $ABCD$ étant un parallélogramme, $\vecteur{DC}=\vecteur{AB}$.

        Pour conclure : $\vecteur{BI}=\vecteur{AB}=\vecteur{DC}=\vecteur{KD}$.

        ~

        \emph{En déduire que $O$ est le milieu de $[BI]$.}

        Puisque $\vecteur{BI}=\vecteur{KD}$, $BIDK$ est un parallélogramme,
        donc ses diagonales $[BD]$ et $[KI]$ se coupent en leur milieu. Le
        point $O$ étant le milieu de $[BD]$, c'est aussi le milieu de $[IK]$.

      \item On utilise la même méthode pour montrer que $O$ est le milieu de $[JC]$.

      \item Nous venons de démontrer que les deux diagonales du quadrilatère $IJKL$ se coupent en leurs milieux. C'est donc un parallélogramme.
  \end{enumerate}
  \end{exercice}

  \begin{exercice}[Algorithmique]~
    \emph{Corrigé en classe.}
  \end{exercice}

  \begin{exercice}[Statistiques]~
    \begin{enumerate}[(1)]
      \item

        \emph{D'après le tableau de fréquences, peut-on dire que les voyelles sont plus utilisées que les consonnes dans la langue française ?}

        Ormi \emph{y}, la voyelle la moins utilisées est \emph{o} (5,2\%).
        Seules cinq consonnes sur 19 \emph{(l, n, r, s, t)} sont utilisées
        davantage. On peut donc dire que les voyelles sont plus utilisées que
        les consonnes.

        D'un autre côté, la somme des fréquences des voyelles est 43,73\%, soit
        moins de la moitié. De ce point de vue là, les voyelles apparaissent
        moins souvent  que les consonnes.

        La question est donc mal formulée : suivant l'interprétation, la
        réponse est différente.


      \item Voici les trois premiers vers d'un poème d'Omar Khayyam
        (mathématicien perse) contenant 132 lettres.

        \begin{tabular}{|l}
          Ma venue ne fut d'aucun profit pour la sphère céleste\\
          Mon départ ne diminuera ni sa beauté ni sa grandeur\\
          Mes deux oreilles n'ont jamais entendu dire par personne\\
        \end{tabular}

        \begin{enumerate}[(a)]
          \item Le \emph{e} a une fréquence d'apparition de 17,69\%. Il devrait donc apparaître environ $132\times17,69\div100=23,4$ fois. Le \emph{z} devrait apparaître $132\times 0,12\div100=0,2$ fois.
          \item Le poème contient 23 \emph{e} (avec ou sans accents), et aucun \emph{z}.
          \item La fréquence d'apparition des lettres dans le poème correspond aux fréquences calculées dans le tableau.
        \end{enumerate}

      \item \emph{Chiffre de César : \og{}VO ZYEBAEYS NO MODDO FOXEO OD MOVES NO MO NOZKBD\fg}

        \begin{enumerate}[(a)]
          \item Voici un tableau présentant l'effectif, l'effectif total et la fréquence des lettres du vers codé (les lettres non représentées n'apparaissent pas).

            \hspace{-0.6cm}\begin{tabular}{c|cccccccc}
          Lettre     & a & b & d & e & f & k & m & n \\
              \hline
          Effectif   & 1 & 2 & 4 & 4 & 1 & 1 & 3 & 3 \\
          Fréquence (\%) & 2,6 & 5,3 & 10,5 & 10,5 & 2,6 & 2,6 & 7,9 & 7,9 \\
        \end{tabular}

        \hspace{-0.6cm}\begin{tabular}{c|cccccc|c}
          Lettre     & o & s & v & x & y & z & Total\\
          \hline
          Effectif   & 11&2 & 2 & 1 & 2 & 1 &  38  \\
          Fréquence (\%) & 28,9&5,3 & 5,3 & 2,6 & 5,3 & 2,6 & 100
        \end{tabular}

          \item La lettre $o$ est celle qui apparait le plus souvent. Il est
            donc probable qu'elle code la lettre $e$. Dans ce cas, les
            lettres $o$, $p$, $q$, etc. codent, respectivement, les lettres
            $f$, $g$, $h$, etc. Le vers décodé donne donc 

        \begin{tabular}{|l}
          Le pourquoi de cette venue et celui de ce départ
        \end{tabular}
      \item \emph{N DHRYYR RCBDHR N IRPH BZNE XUNLLNZ ?}
        En utilisant la même méthode, on trouve que les lettres les plus représentées sont les lettres $r$ et $n$. Il est donc probable que l'une de ces lettres code la lettre $e$.

        \begin{itemize}
          \item Si $n$ code le $e$, alors $n$, $o$, $p$, etc. codent les $e$, $f$, $g$ et la phrase devient : \og{}E UYIPPI ITSUYI E ZIGY SQEV OLECCEQ\fg, ce qui ne signifie pas grand'chose.
          \item Si $r$ code le $e$, alors $r$, $s$, $t$, etc. codent les $e$, $f$, $g$ et la phrase devient : \og{}A QUELLE EPOQUE A VECU OMAR KHAYYAM\fg. La réponse est aux alentours de 1100 (né en 1048, mort en 1131).
        \end{itemize}
        \end{enumerate}


    \end{enumerate}
  \end{exercice}

\end{document}
