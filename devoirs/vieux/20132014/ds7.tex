\documentclass[11pt]{article}

\usepackage{pablo}
\usepackage{pablo-listings}
\usepackage[a5paper,margin=2cm]{geometry}
\usepackage{eurosym}
\pagestyle{empty}

\begin{document}

\begin{center}
  Devoir surveillé --- 1h

  \textsc{Systèmes et Droites}
\end{center}
Nom : \ldots\ldots\ldots\ldots\ldots\ldots\ldots\ldots\ldots\ldots

\begin{exercice}[Équations de droites --- 5,5 points] \emph{Les lectures graphiques doivent apparaître clairement sur les graphiques. Les résultats seront donnés par des valeurs exactes.} 
 
  \begin{center}
\begin{tikzpicture}[scale=0.8]
\shorthandoff{:}
    % Quadrillage
    \draw[color=black,step=1,dotted] (-5.5,-5.5) grid (7.5,4.5);

    % Repère
    \draw (-5.25,0) -- (7,0);
    \draw[thick,->] (0,0) -> (1,0) node[midway,below] {$\vecteur\imath$};
    \draw (0,-5.25) -- (0,4);
    \draw[thick,->] (0,0) -> (0,1) node[midway,left] {$\vecteur\jmath$};
    \draw (0,0) node[below left] {$O$};

%     % Objets
      \draw[smooth,samples=100,domain=-0.25:4.75] plot(\x,{-2*(\x)+4}) node[right] {$d_1$};
      \draw[smooth,samples=100,domain=-5.5:7.5] plot(\x,{1/3*(\x)-5/3}) node[right] {$d_2$};
      \draw (-3,-5.5) -- (-3,4.5) node[right] {$d_3$} ;
  \end{tikzpicture}
\end{center}

\begin{enumerate} 
 \item Donner les équations des droites $d_1$, $d_2$ et $d_3$. 
\item Déterminer les coordonnées du point d'intersection de $d_1$ et $d_2$. 
\item Tracer, sur le graphique précédent,
les droites d'équations $y = -\frac{1}{2}x+3$ et $x = 6$. 
\end{enumerate}
%\begin{center}
%  \begin{tikzpicture}[scale=0.5]
%    \shorthandoff{:}
%    % Quadrillage
%    \draw[color=black,step=1,dotted] (-5.5,-5.5) grid (7.5,4.5);
%
%    % Repère
%    \draw (-5.25,0) -- (7,0);
%    \draw[thick,->] (0,0) -> (1,0) node[midway,below] {$\vecteur\imath$};
%    \draw (0,-5.25) -- (0,4);
%    \draw[thick,->] (0,0) -> (0,1) node[midway,left] {$\vecteur\jmath$};
%    \draw (0,0) node[below left] {$O$};
%
%    % Solutions
%    \draw[smooth,samples=100,domain=-3:7.5] plot(\x,{-1/2*(\x)+3});
%    \draw (6,-5.5) -- (6,4.5);
%
%  \end{tikzpicture}
%\end{center}
 
\end{exercice}

\newpage

\begin{exercice}[Vecteurs et droites --- 6 points]\emph{Les réponses par lecture graphique ne seront pas prises en compte, sauf quand l'énoncé l'autorise explicitement.}

  \begin{center}
    \begin{tikzpicture}[scale=0.9]
    % Quadrillage
    \draw[color=black,step=1,dotted] (-.5,-.2) grid (7.5,4.2);

    % Repère
    \draw (-.25,0) -- (7,0);
    \draw[thick,->] (0,0) -> (1,0) node[midway,below] {$\vecteur\imath$};
    \draw (0,-.25) -- (0,4);
    \draw[thick,->] (0,0) -> (0,1) node[midway,left] {$\vecteur\jmath$};
    \draw (0,0) node[below left] {$O$};

    % Objets
    \draw (1,1) node{$\bullet$}node{$\bullet$}  node[right] {$A$};
    \draw[color=blue,->] (2,4) -- (6,4) node[midway,below]{$\vecteur{u}$};
    \draw[color=blue,->] (5,1) -- (1,3) node[midway,above]{$\vecteur{v}$};

    % Solutions
    %\draw[color=blue,->] (4,1) -- (6,2) node[midway,above]{$\vecteur{w}$};
    %\draw[color=red,domain=-0.1:7] plot (\x,{\x/2+.5}) node[below left]{$\cal D$};
  \end{tikzpicture}
\end{center}

  \begin{enumerate}
    \item Lire sur le graphique les coordonnées du point $A$ et des vecteurs $\vecteur{u}$ et $\vecteur{v}$.
    \item Soit le vecteur $\vecteur w$, défini par $\vecteur w=\vecteur u +\frac{1}{2}\vecteur v$. Montrer que les coordonnées de $\vecteur w$ sont $\coord{2}{1}$.
    \item \begin{enumerate}
        \item Tracer la droite $\cal D$ de vecteur directeur $\vecteur w$ passant par $A$.
        \item Le point $B(3,2;2,5)$ appartient-il à la droite $\cal D$ ?
          Vérifier votre réponse sur le graphique.
      \end{enumerate}
    \item \begin{enumerate}
        \item
          Que fait l'algorithme suivant ?
      \begin{lstlisting}[language=naturel,frame=lines,mathescape=true]
      Lire x
      Lire y
      Si 0,5 * x + 0,5 = y
      Alors
        Afficher "Vrai"
      Sinon
        Afficher "Faux"
      FinSi
      \end{lstlisting}
    \item On exécute cet algorithme avec $x=3$ et $y=2$. Qu'affiche l'algorithme ? Que peut-on en déduire ?
  \end{enumerate}
  \end{enumerate}
\end{exercice}

\newpage

\begin{exercice}[Système d'équations --- 5 points]\emph{Toutes les questions,
  sauf la dernière, peuvent être résolues sans avoir résolu les précédentes.}
  \begin{enumerate}
    \item Résoudre les systèmes suivants :

      \hspace{\stretch{1}}
      \begin{inparaenum}
      \item $\systeme{y=-5-x}{y=\frac{1}{2}x+1}$
        \hspace{\stretch{1}}
      \item $\systeme{2y=x}{y=\frac{1}{2}x+1}$
        \hspace{\stretch{1}}
      \end{inparaenum}
      \hspace{\stretch{1}}
      ~

    \item Le but de la question est de résoudre le système $(S)$ suivant :
      \[(S)\systeme{x^2+5x-xy-10y-2y^2=0}{y=\frac{1}{2}x+1}\]
      \begin{enumerate}
        \item Montrer que $(S)$ est équivalent à :

          \[(S)\systeme{(x-2y)(5+x+y)=0}{y=\frac{1}{2}x+1}\]
        \item Montrer que $(S)$ est équivalent à :

          \[\systeme{2y=x}{y=\frac{1}{2}x+1} \text{ ou } \systeme{y=-5-x}{y=\frac{1}{2}x+1}\]
        \item En utilisant la première question, en déduire les solutions de $(S)$.
      \end{enumerate}
  \end{enumerate}
\end{exercice}

\begin{exercice}[Problème ouvert --- 3,5 points]\emph{Cet exercice est ouvert ; toute trace de recherche sera valorisée.}

 Deux entreprises A et B emploient deux types de personnel : des cadres et des ouvriers. 
\begin{itemize} 
 \item L'entreprise A emploie 5 cadres et 20 ouvriers. Le salaire moyen des cadres est 3020~\officialeuro{} et celui des ouvriers 1750~\officialeuro.
\item  L'entreprise B emploie 50 personnes. Le salaire moyen des cadres est 2880~\officialeuro{} et celui des ouvriers 1650~\officialeuro.
\end{itemize}
 Le directeur financier de l'entreprise B affirme que le salaire moyen pour l'ensemble de ses employés est supérieur à celui de l'entreprise A. Est-ce possible ? 
\end{exercice}

\end{document}

