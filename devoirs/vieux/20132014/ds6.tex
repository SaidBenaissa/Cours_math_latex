\documentclass[11pt]{article}

\usepackage{pablo}
\usepackage{pablo-listings}
\usepackage[a5paper,margin=1.8cm]{geometry}
\usepackage{multicol}

\pagestyle{empty}
\begin{document}

\begin{center}
  {\large
    Devoir surveillé --- 1h

    \textsc{Équations --- Repérage}
  }
\end{center}

\begin{exercice}[(In)équations --- 3 points]~
  \begin{enumerate}
    \item Résoudre l'équation :
          $\dfrac{2+x}{x-3}=0$
    \item Résoudre l'inéquation :
      $(2x-10)(x-1)\geq0$
  \end{enumerate}
\end{exercice}

\begin{exercice}[Développement, factorisation --- 6,5 points]~

  Soit $A(x)=(x+1)^2-(x+1)(2x-4)$, avec $x\in{\mathbb R}$.
  \begin{enumerate}
    \item Développer, réduire et ordonner $A(x)$.
    \item Factorise $A(x)$.
    \item Choisir la forme la plus adaptée pour résoudre dans $\mathbb R$ les équations suivantes.

      \begin{inparaenum}
      \item $A(x)=0$
        \hspace{\stretch{1}}
      \item $A(x)=5$
        \hspace{\stretch{1}}
      \item $A(x)=(x+1)^2$
      \end{inparaenum}
  \end{enumerate}
\end{exercice}

\begin{exercice}[Repérage --- 7 points]~

  Soient les points $A(-1;1)$, $B(2;3)$, $C(4;2)$, $D(1;0)$.
  \begin{enumerate}
    \item Placer ces points dans un repère orthonormé.
    \item On va déterminer de deux manières différentes la nature du quadrilatère $ABCD$.
      \begin{description}
        \item[(a) Première méthode] Calculer les coordonnées des vecteurs $\vecteur{AB}$ et $\vecteur{DC}$. En déduire la nature du quadrilatère $ABCD$.
        \item[(b) Seconde méthode] Calculer les coordonnées du milieu du segment $AC$, et celles du milieu du segment $BD$. En déduire la nature du quadrilatère $ABCD$.
      \end{description}
    \item $ABCD$ est-il un losange ? Justifier (sans lecture graphique).
  \end{enumerate}
\end{exercice}

\vfill\hfill
\emph{Tourner la page.}
\newpage

\begin{exercice}[Repérage, algorithmique --- 3,5 points]~
  \begin{enumerate}
    \item Rappeler la formule permettant de calculer la longueur du segment $AB$, avec $A(x_A,y_A)$ et $B(x_B,y_B)$.
    \item On considère l'algorithme suivant.
      \begin{lstlisting}[language=naturel,frame=lines,mathescape=true]
      Lire $x_A$
      Lire $y_A$
      Lire $x_B$
      Lire $y_B$
      Lire $x_C$
      Lire $y_C$
      $\sqrt{(x_B-x_A)^2+(y_B-y_A)^2}$ -> $AB$
      $\sqrt{(x_C-x_A)^2+(y_C-y_A)^2}$ -> $AC$
      $\sqrt{(x_C-x_B)^2+(y_C-y_B)^2}$ -> $BC$
      Si $AB^2+AC^2=BC^2$
      Alors
        Afficher "Vrai"
      Sinon
        Afficher "Faux"
      FinSi
      \end{lstlisting}
      \begin{enumerate}
        \item Que fait cet algorithme ?
        \item Modifier cet algorithme pour qu'il détermine si le triangle $ABC$ est isocèle en $A$.
      \end{enumerate}
  \end{enumerate}

\end{exercice}

\end{document}
