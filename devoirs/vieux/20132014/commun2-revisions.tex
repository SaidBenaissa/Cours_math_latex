\documentclass[10pt]{article}

\usepackage{pablo}
\usepackage[a6paper,margin=0.9cm]{geometry}
\pagestyle{empty}

\begin{document}

\begin{center}
  \textsc{Devoir commun}

  Le jeudi 22 mai de 11h à 12h, en salle 124
\end{center}

\paragraph{Vecteurs et Repérage}

\begin{compactitem}
  \item Tout le chapitre 2.
  \item Savoir : Formule des coordonnées d'un vecteur $\vecteur{AB}$ ; Formule de la longueur d'un segment $[AB]$ ; Formule des coordonnées du milieu d'un segment $[AB]$.
  \item Savoir faire : Manipulation des coordonnées de vecteurs (somme, multiplication) ; Utilisation des formules ci-dessus. \emph{Ex. 45, 58, 60 p. 174-175.}
  \item Déterminer si deux vecteurs sont colinéaires. \emph{Ex. 54 p. 175}
\end{compactitem}

\paragraph{Systèmes linéaires --- Équations de droites}
\begin{compactitem}
  \item Tracer une droite dont on donne une équation ; Tracer une droite définie par un point et un vecteur directeur. \emph{Ex. 5, 23 p.202-203}
  \item Vérifier algébriquement si un point appartient à une droite. \emph{Ex. 11 p.202}
  \item Lire graphiquement l'équation d'une droite ; Calculer l'équation d'une droite définie par deux points dont on connait les coordonnées. \emph{Ex. 1 et 18 p. 201-202}
  \item Déterminer algébriquement le point d'intersection de deux droites (résolution d'un système). \emph{Ex. 28, 47 p. 205}
  \item Résoudre un système de deux équations à deux inconnues. \emph{36 à 40 p. 204}
\end{compactitem}


\paragraph{Fonctions}
\begin{compactitem}
  \item Définitions et méthodes du chapitre 1 : image, antécédent, domaine de définition, tableau de variation, lecture graphique, etc.
  \item Fonctions linéaires et affines.
  \item Résoudre une inéquation en utilisant un tableau de signes. \emph{Ex. 29 p. 141}.
\end{compactitem}

\paragraph{Fonction carrée --- Trinômes}
\emph{Pour plus de détails, voir la feuille de révision du devoir du 17 février.}

\begin{compactenum}
\item \emph{Variations et extremums}
    Pour chacune des fonctions suivantes :
    \begin{compactenum}
      \item dresser son tableau de variations ;
      \item déterminer les coordonnées de son extremum ;
      \item tracer sa représentation graphique et vérifier graphiquement les réponses aux questions précédentes.
    \end{compactenum}
    \begin{inparaenum}[(a)]
    \item $2x^2-3x-1$ ;
    \item $-5x^2+2$ ;
    \item $3x^2$ ;
    \item $-x^2+6x+1$.
    \end{inparaenum}
  \item \emph{Forme canonique}
  \begin{compactenum}
    \item
      \begin{compactenum}
        \item Montrer que $-(x-0,5)^2-3,75=-x^2+x-4$.
        \item En déduire les solutions de $-x^2+x-4=0$.
      \end{compactenum}
    \item Même question avec $(x+7)^2-9$ et $x^2+14x+40$.
    \item Même question avec $-3(x+\frac{1}{3})^2+9$ et $-3x^2+2x+\frac{26}{3}$.
  \end{compactenum}
\end{compactenum}

\paragraph{Pas au programme du devoir}

Algorithmique ; Espace ; Probabilités ; Statistiques ; Fonction inverse et homographies

\end{document}

