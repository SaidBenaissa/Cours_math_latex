\documentclass[12pt]{article}

\usepackage{pablo}
\usepackage{pablo-listings}
\usepackage[a5paper,margin=1.5cm]{geometry}
\usepackage{enumerate}
\usepackage{multicol}
\usepackage{framed}

\pagestyle{empty}
\begin{document}

  \section*{Devoir à la maison}
  \emph{À rendre le vendredi 8 novembre}
  \setcounter{exercice}{0}

  \begin{exercice}[Vecteurs]
    Exercice 90 page 178.
  \end{exercice}

  \begin{exercice}[Algorithmique]~
    \begin{enumerate}[(1)]
      \item On considère l'algorithme suivant.
        \begin{lstlisting}[literate={->}{{$\rightarrow$}}1 {*}{$\times$}1 ,frame=simple]
Choisir un nombre n
n -> m
n + 3 -> n
n * 2 -> n
n + m -> n
n / 3 -> n
n - 2 -> n
Afficher n
        \end{lstlisting}
        \begin{enumerate}[(a)]
          \item Exécuter (sans la calculatrice) l'algorithme avec 1, 5, puis
            10 comme nombre de départ.
          \item Que fait l'algorithme ?
        \end{enumerate}
      \item Un photographe propose le tarif suivant pour développer des
        photographies : 0,16 euros par photo pour moins de 75 photos ; 3 euros
        plus 0,12 euros par photo pour 75 photos et plus.

        Écrire un algorithme qui prend en entrée le nombre de photo à
        développer, et qui affiche le prix total.
    \end{enumerate}
  \end{exercice}

  \begin{exercice}[Statistiques]~
    \begin{enumerate}[(1)]
      \item Dans la langue française, la fréquence, en pourcentage, d'apparition des différentes lettres de l'alphabet est la suivante.

        \hspace{-1cm}\begin{tabular}{c|c|c|c|c|c|c|c|c|c}
          Lettre&a&b&c&d&e&f&g&h&i\\\hline
          Fréquence & 8,11 & 0,81 & 3,38 & 4,28 & 17,69 & 1,13 & 1,19 & 0,74 & 7,24 \\
        \end{tabular}

        \hspace{-1cm}\begin{tabular}{c|c|c|c|c|c|c|c|c|c}
          Lettre&j&k&l&m&n&o&p&q&r\\\hline
          Fréquence & 0,18 & 0,02 & 5,99 & 2,29 & 7,68 & 5,2 & 2,92 & 0,83 & 6,53 \\
        \end{tabular}

        \hspace{-1cm}\begin{tabular}{c|c|c|c|c|c|c|c|c|c}
          Lettre&s&t&u&v&w&x&y&z\\
          \hline
          Fréquence & 8,87 & 7,44 & 5,23 & 1,28 & 0,06 & 0,53 & 0,26 & 0,12
        \end{tabular}

        D'après ce tableau de fréquences, peut-on dire que les voyelles sont plus utilisées que les consonnes dans la langue française ?

      \item Voici les trois premiers vers d'un poème d'Omar Khayyam
        (mathématicien perse) contenant 132 lettres.

        \begin{tabular}{|l}
          Ma venue ne fut d'aucun profit pour la sphère céleste\\
          Mon départ ne diminuera ni sa beauté ni sa grandeur\\
          Mes deux oreilles n'ont jamais entendu dire par personne\\
        \end{tabular}

        \begin{enumerate}[(a)]
          \item D'après le tableau de la question précédente, quel devrait être le nombre de lettres \og{}e\fg{} et le nombre de lettres \og{}z\fg{} dans ce poème ?
          \item Combien de lettres \og{}e\fg{} et \og{}z\fg{} le poème contient-il ?
          \item Expliquer les résultats obtenus.
        \end{enumerate}

      \item Le codage d'un texte selon le \emph{chiffre de César} correspond
        au remplacement de chaque lettre par une autre située $n$ places plus
        loin dans l'alphabet. Par exemple, si $n=1$, les lettres $a$ sont
        remplacées par $b$, les $b$ par $c$, etc.

        Le quatrième vers du poème d'Omar Khayyam, codé selon cette méthode, est : « VO ZYEBAEYS NO MODDO FOXEO OD MOVES NO MO NOZKBD ».

        \begin{enumerate}[(a)]
          \item Établir la fréquence d'apparition des lettres de ce texte codé.
          \item En supposant que ces fréquences sont similaires à celles données à la première question, décoder le vers.
          \item N DHRYYR RCBDHR N IRPH BZNE XUNLLNZ ?
        \end{enumerate}


    \end{enumerate}
  \end{exercice}

\end{document}
