\documentclass[11pt]{article}

\usepackage{pablo}
\usepackage[a5paper,margin=1.2cm]{geometry}
\usepackage{enumerate}

\usepackage{multicol}

\pagestyle{empty}
\begin{document}

\begin{center}
  {\large
    Devoir surveillé
    ---
    \textsc{Vecteurs}
  }
\end{center}

\begin{exercice}[Relation de Chasles --- 3 points] Exprimer le plus
  simplement possible les sommes de vecteurs suivants :
  \begin{multicols}{2}
    \begin{enumerate}[(a)]
      \item $\vecteur{AB}+\vecteur{BC}$
      \item $\vecteur{EF}+\vecteur{AE}-\vecteur{AF}$
      \item $\vecteur{AB}+\vecteur{CA}-\vecteur{DB}-\vecteur{CD}$
      \item $2\vecteur{DB}+\vecteur{BD}-\vecteur{AB}$
    \end{enumerate}
  \end{multicols}
\end{exercice}

\begin{exercice}[2 points]
  Exprimer les vecteur $\vecteur u$ et $\vecteur v$ en fonction des vecteurs
  $\vecteur a$ et $\vecteur b$.

  \begin{multicols}{2}
    \begin{enumerate}[(a)]
      \item

        \tikzstyle{vecteur}=[thick]
        \begin{tikzpicture}[baseline=(current bounding box.north),scale=0.7]
          \draw[thin,color=gray,step=1] (-1,-1) grid (3,4);
          \draw[vecteur,->] (0,0) -- (2,0);
          \draw[vecteur,->] (0,0) -- (0,1);
          \draw[vecteur,->] (0,0) -- (2,3);
          \draw (0,0.5) node[left] {$\vecteur{a}$};
          \draw (1.5,0) node[below] {$\vecteur{b}$};
          \draw (1.5,1.5) node[] {$\vecteur{u}$};
        \end{tikzpicture}

      \item
        \begin{tikzpicture}[baseline=(current bounding box.north),scale=0.7]
          \draw[thin,color=gray,step=1] (-1,-3) grid (6,2);
          \draw[vecteur,->] (0,0) -- (2,0);
          \draw[vecteur,->] (0,0) -- (0,1);
          \draw[vecteur,->] (0,0) -- (5,-2);
          \draw (0,0.5) node[left] {$\vecteur{a}$};
          \draw (1.5,0) node[above] {$\vecteur{b}$};
          \draw (2.5,-1.5) node[] {$\vecteur{v}$};
        \end{tikzpicture}
    \end{enumerate}
  \end{multicols}

\end{exercice}

\begin{exercice}[5 points]
  Soit $ABCD$ un quadrilatère quelconque, et $I$, $J$, $K$, $L$ les milieux respectifs des segments $[AB]$, $[BC]$, $[CD]$, et $[DA]$.
  \begin{enumerate}[(a)]
    \item Sur une figure, placer quatre points $A$, $B$, $C$ et $D$ au
      hasard, et placer les points $I$, $J$, $K$ et $L$ en fonction de $A$,
      $B$, $C$ et $D$. Tracer les quadrilatères $ABCD$ et $IJKL$, et le
      segment $AC$.
    \item Le but de cette question est de prouver que $\vecteur{IJ}=\frac{1}{2}\vecteur{AC}$.
      \begin{enumerate}[(i)]
        \item Justifier que $\vecteur{IJ} = \vecteur{IB}+\vecteur{BJ}$.
        \item Justifier que $\vecteur{IB}=\vecteur{AI}$ et $\vecteur{BJ}=\vecteur{JC}$.
        \item En déduire que $\vecteur{AI}+\vecteur{JC}=\vecteur{IJ}$.
        \item Justifier que $\vecteur{AC}=\vecteur{AI}+\vecteur{IJ}+\vecteur{JC}$.
        \item En déduire que $\vecteur{IJ}=\frac{1}{2}\vecteur{AC}$.
      \end{enumerate}
    \item En utilisant le même raisonnement, montrer que $\vecteur{LK}=\frac{1}{2}\vecteur{AC}$.
    \item En déduire la nature du quadrilatère $IJKL$.
  \end{enumerate}
\end{exercice}

\begin{exercice}[Bonus]
  Citer un mathématicien, et dire pourquoi il est connu.

  \begin{em}
    \noindent
    Barème : 0,5 point pour la justesse de la réponse, plus 0,5 point pour
    l'originalité (si aucun autre élève n'a donné la même réponse).
  \end{em}
\end{exercice}

\end{document}
