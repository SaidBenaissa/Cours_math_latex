\documentclass[10pt]{article}

\usepackage{pablo}
\usepackage[a6paper,margin=0.5cm]{geometry}
\pagestyle{empty}

\begin{document}

\begin{center}
  \textsc{Devoir du 14 avril} --- Révisions
\end{center}

\paragraph{Trinômes}
  Voir la feuille de révision du devoir précédent.

\paragraph{Espace}

\emph{Pages 251 et suivantes.}

\subparagraph{Perspective cavalière}

\begin{compactitem}
  \item Savoir lire un dessin en perspective cavalière. \dotfill 1 et 2
  \item Savoir dessiner une figure en perspective cavalière. \dotfill 2
\end{compactitem}

\subparagraph{Solides usuels}

\begin{compactitem}
  \item Utiliser les formules d'aire et de volume. \dotfill 7, 8
  \item Résoudre des problèmes en extrayant des figures planes. \dotfill 6, 10, 11
\end{compactitem}

\paragraph{Probabilités}

\emph{Pages 207 et suivantes.}

\begin{compactitem}
  \item Connaître les définitions : expérience aléatoire, issue, évènement élémentaire, évènement, etc.
  \item Calculer des probabilités dans des cas d'équiprobabilité \dotfill 9, 11, 13, 16
  \item Savoir manipuler le contraire d'un évènement \dotfill 21, 27
  \item Savoir décrire des union, intersection et contraire d'évènements \dotfill \emph{voir point suivant}
  \item Savoir manipuler les union et intersection d'évènements. \hfill 25, 27, 28, 30
  \item Savoir reconnaiter et utiliser des évènements incompatibles. \dotfill 27, 28
  \item Utiliser un tableau pour résoudre un problème. \dotfill 25, 26
  \item Utiliser un arbre pour résoudre un problème. \dotfill 16, 18, 20
\end{compactitem}

\end{document}

