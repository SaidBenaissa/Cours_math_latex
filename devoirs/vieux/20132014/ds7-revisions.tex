\documentclass[10pt]{article}

\usepackage{pablo}
\usepackage[a6paper,margin=1cm]{geometry}
\pagestyle{empty}

\begin{document}

\begin{center}
  Révisions pour le devoir du \hspace{1cm} février

  \textsc{Systèmes et Droites}
\end{center}

\paragraph{Repérage}
\begin{description}
  \item[Savoir] Formule des coordonnées d'un vecteur $\vecteur{AB}$ ; Formule de la longueur d'un segment $[AB]$ ; Formule des coordonnées du milieu d'un segment $[AB]$.
  \item[Savoir faire] Manipulation des coordonnées de vecteurs (somme, multiplication) ; Utilisation des formules ci-dessus. \emph{Ex. 45, 58, 60 p. 174-175.}
\end{description}

\paragraph{Système et droites}
\begin{description}
  \item[Savoir faire] Tracer une droite dont on donne une équation ; Tracer une droite définie par un point et un vecteur directeur. \emph{Ex. 5, 23 p.202-203}
  \item[Savoir faire] Vérifier algébriquement si un point appartient à une droite. \emph{Ex. 11 p.202}
  \item[Savoir faire] Lire graphiquement l'équation d'une droite ; Calculer l'équation d'une droite définie par deux points dont on connait les coordonnées. \emph{Ex. 1 et 18 p. 201-202}
  \item[Savoir faire] Déterminer algébriquement le point d'intersection de deux droites (résolution d'un système). \emph{Ex. 28, 47 p. 205}
  \item[Savoir faire] Résoudre un système de deux équations à deux inconnues. \emph{36 à 40 p. 204}
\end{description}

\end{document}

