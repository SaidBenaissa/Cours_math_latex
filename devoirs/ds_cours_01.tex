\documentclass[14pt, aspectratio=169]{beamer}

\usepackage[utf8]{inputenc}
\usepackage[T1]{fontenc}
\usepackage[francais]{babel}

\useoutertheme{infolines}
\usecolortheme{crane}

\institute{Lycée Marie Curie}
\date{2014}
\logo{}
\title{Repérage}

\begin{document}


\section{Contrôle de cours}
\begin{frame}
\end{frame}

\begin{frame}
  On considère les points $A\left( x_A,y_A \right)$ et $B\left( x_B,y_B \right)$.
  \begin{enumerate}
    \item À quelles conditions un repère $(O,I,J)$ est-il \emph{orthonormé} ?
    \item Quelles sont les coordonnées du milieu $I$ du segment $[AB]$ ?
    \item Quelle est la longueur du segment $[AB]$ ?
    \item Donner trois conditions équivalentes à « $ABCD$ est un parallélogramme ».
  \end{enumerate}
\end{frame}

\end{document}
