\documentclass{article}

\usepackage{pablo-devoir}
\usepackage{pablo-listings}
\usepackage[a5paper]{geometry}

\pagestyle{empty}

\title{Repérage}
\date{Mardi 23 septembre}
\classe{2\up{des}14}
\dsnum{DM 1}

\begin{document}

\maketitle
\begin{exercice}[Lieu géométrique]
  Dans un repère orthonormé, on considère un point $A(0;6)$, le cercle $\mathcal{C}$ de centre $A$ et de rayon 5, et la droite $d$ d'équation $y=3$. On recherche les coordonnées des points d'intersection entre $d$ et $\mathcal{C}$.
  \begin{enumerate}
    \item Placer $A$, $d$ et $\mathcal{C}$ sur une figure.
    \item Soit $B$ un point d'intersection entre $d$ et $\mathcal{C}$. On appelle $x$ son abscisse ; quelle est sont ordonnée ?
    \item Montrer que l'on a : $5=\sqrt{x^2+9}$.
    \item Résoudre l'équation, et conclure.
  \end{enumerate}
\end{exercice}

\begin{exercice}[Algorithmique]
  On considère l'algorithme suivant.
  \begin{lstlisting}[numbers=left,language=naturel,frame=lines,mathescape=true,escapeinside={(*@}{@*)}]
  Lire AB
  Lire AC
  Lire BC
  Si $\cdots$ (*@\label{todo}@*)
  Alors
    Afficher "Le triangle est rectangle en A"
  Sinon
    Afficher "Le triangle n'est pas rectangle en A"
  FinSi
  \end{lstlisting}

  Compléter la ligne \ref{todo} de l'algorithme pour qu'étant donné les trois longueurs d'un triangle, il détermine si oui ou non ce triangle est rectangle en $A$.

\end{exercice}

\end{document}
