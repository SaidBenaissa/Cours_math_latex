\documentclass{article}

\usepackage{pablo}
\usepackage[a5paper]{geometry}
\pagestyle{empty}

\begin{document}

\begin{center}
  \textbf{Algorithmique et Repérage}
\end{center}

\begin{exercice}
  Écrire un algorithme qui lit les coordonnées de deux points $A$ et $B$, et qui affiche les coordonnées du milieu du segment $[AB]$.
\end{exercice}

\begin{exercice}
  Écrire un algorithme qui lit les coordonnées de deux points $A$ et $B$ (dans un repère orthonormé), et qui affiche la longueur du segment $[AB]$.
\end{exercice}

\begin{exercice}
Écrire un algorithme qui prend en argument les longueurs de trois segments $AB$, $BC$ et $AC$, et qui détermine si le triangle $ABC$ est équilatéral ou non.
\end{exercice}

\begin{exercice}
  Écrire un algorithme qui prend en argument les coordonnées de trois points $A$, $B$ et $I$, et qui détermine si le point $I$ est le milieu du segment $[AB]$.
\end{exercice}

\begin{exercice}
Écrire un algorithme qui prend en argument les coordonnées de deux points $A$ et $B$, et une longueur $r$, et qui détermine si le point $B$ appartient au cercle de centre $A$ et de rayon $r$.
\end{exercice}

\begin{exercice}
Écrire un algorithme qui prend en argument les coordonnées de trois points $A$, $B$, $C$, et qui détermine si le triangle $ABC$ est équilatéral ou non.
\end{exercice}

\begin{exercice}
Écrire un algorithme qui prend en argument les coordonnées de quatre points $A$, $B$, $C$ et $D$, et qui affiche $Vrai$ si $ABCD$ est un parallélogramme.
\end{exercice}

\begin{exercice}
Écrire un algorithme qui prend en argument les coordonnées de trois points $A$, $B$ et $C$, et qui affiche les coordonnées d'un point $D$ tel que $ABCD$ soit un parallélogramme.
\end{exercice}

\end{document}
