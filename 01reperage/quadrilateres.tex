\documentclass[twocolumn,12pt]{article}

\usepackage{pablo}
\usepackage[landscape,a5paper,margin=1cm]{geometry}
\pagestyle{empty}

\begin{document}

\begin{propriete}[Losange]
  Les propositions suivantes sont équivalentes.
  \begin{enumerate}[(a)]
    \item $ABCD$ est un losange.
    \item $ABCD$ est un parallélogramme dont les diagonales sont perpendiculaires.
    \item $ABCD$ est un parallélogramme ayant deux côtés consécutifs de même longueur.
    \item $ABCD$ est un quadrilatère ayant quatre côtés de même longueur.
  \end{enumerate}
\end{propriete}

\begin{propriete}[Rectangle]
  Les propositions suivantes sont équivalentes.
  \begin{enumerate}[(a)]
    \item $ABCD$ est un rectangle.
    \item $ABCD$ est un parallélogramme ayant un angle droit.
    \item $ABCD$ est un parallélogramme dont les diagonales sont de même longueur.
    \item $ABCD$ est un quadrilatère ayant quatre angles droits.
  \end{enumerate}
\end{propriete}

\begin{definition}[Carré]
  Un quadrilatère qui est à la fois un losange et un rectangle est appelé un \emph{carré}.
\end{definition}

\begin{center}
  \begin{tikzpicture}[ultra thick,scale=2.3]
        \tikzstyle{propriete} = []
        \tikzstyle{fleche} = [-latex,thick,shorten >=15pt,shorten <=15pt]

  \draw (0,0) node(quadri){Quadrilatère};
  \draw ($(quadri.north east) + (0,.3)$) -- ($(quadri.north west)+(.1,0)$) -- ($(quadri.south west)+(-.3,-.1)$) -- ($(quadri.south east)+(.1,0)$) -- cycle;

  \draw (0,-1.5) node(para){Parallélogramme};
  \draw ($(para.north east) + (0,.2)$) -- ($(para.north west)+(-.4,0)$) -- ($(para.south west)+(0,-.2)$) -- ($(para.south east)+(.4,0)$) -- cycle;

  \draw(-1,-3) node(rectangle){Rectangle};
  \draw (rectangle.north east) -- (rectangle.north west) -- (rectangle.south west) -- (rectangle.south east) -- cycle;

  \draw(1,-3) node(losange){Losange};
  \draw ($(losange.north)+(0,.1)$) -- ($(losange.west)+(-.3,0)$) -- ($(losange.south)+(0,-.1)$) -- ($(losange.east)+(.3,0)$) -- cycle;

  \draw(0,-4.5) node(carre){Carré};
  \draw ($(carre)+(.25,.25)$) -- ($(carre)+(-.25,.25)$) -- ($(carre)+(-.25,-.25)$) -- ($(carre)+(.25,-.25)$) -- cycle;


  \draw[fleche] (quadri) -- (para) node[midway,right,propriete]{\begin{tabular}{c}Côtés parallèles\\deux à deux\end{tabular}};
  \draw[fleche] (para) -- (rectangle) node[midway,left,propriete]{Un angle droit};
  \draw[fleche] (para) -- (losange) node[midway,right,propriete]{\begin{tabular}{c}Côtés consécutifs\\de même longueur\end{tabular}};
  \draw[fleche] (losange) -- (carre);
  \draw[fleche] (rectangle) -- (carre);
\end{tikzpicture}
\end{center}

\end{document}
