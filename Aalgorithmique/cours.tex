\section{Algorithme}

\begin{definition}
  Un \emph{algorithme} est une suite finie d'opérations élémentaires permettant
  de résoudre un problème donné.
\end{definition}

\begin{exemple}
  \ldots
\end{exemple}

Trois parties (qui peuvent être mélangées) composent un algorithme :
\begin{itemize}
  \item entrée des données ;
  \item traitement des données ;
  \item sortie des résultats.
\end{itemize}

\section{Variables}

\begin{definition}
  Une \emph{variable} est un espace en mémoire, portant un nom, dans lequel on
  peut stocker une valeur.
\end{definition}

L'instruction \texttt{a$\rightarrow$b} se lit \emph{affecter \texttt{a} à
\texttt{b}}, et signifie : aller chercher en mémoire la valeur de \verb+a+, et
placer cette valeur dans \verb+b+. D'autres notations possibles sont :
  \texttt{b $\leftarrow$ a} (« \verb+b+ reçoit \verb+a+) ;
  \texttt{Affecter a à b} ;
  \texttt{a = b} ;
  etc.


\begin{exemple}~
  \begin{center}
  \begin{tabular}{p{8em}|cc}
    Instructions & Contenu de $x$ & Contenu de $y$ \\
    \hline
    \emph{Début} &   &   \\
    \texttt{3$\rightarrow$x} &   &  \\
    \texttt{3x$^2$ + 2x$\rightarrow$y} &   & \\
    \texttt{2y$\rightarrow$y} & &
  \end{tabular}
\end{center}
\end{exemple}

\section{Entrées / Sorties}

Il est possible de demander à l'utilisateur d'affecter une valeur à une
variable, avec une instruction du type \texttt{Lire a}. Le programme attend que
l'utilisateur saisisse une valeur au clavier, puis place dans l'espace mémoire
correspondant à \texttt{a} cette valeur.

Il est possible d'afficher quelque chose à l'écran avec l'instruction
\texttt{Afficher a}.

\begin{exemple}Que fait l'algorithme suivant ?
  \begin{lstlisting}[language=naturel,frame=lines]
  Lire a
  a * a -> a
  Afficher a
  \end{lstlisting}
\end{exemple}

\begin{multicols}{2}
\begin{exercice}
  Écrire un algorithme qui demande un nombre à l'utilisateur,
  calcule l'image de ce nombre par la fonction $f:x\mapsto2x^2-1$, et affiche
  ce rélultat.

  ~

  \hrule
  \vspace{2cm}
  \hrule
\end{exercice}

\columnbreak

\begin{exercice}Que fait l'algorithme suivant ?
  \begin{lstlisting}[language=naturel,frame=lines]
  Lire n
  n -> m
  n + 3 -> n
  n * 2 -> n
  n + m -> n
  n / 3 -> n
  n - 2 -> n
  Afficher n
  \end{lstlisting}
\end{exercice}
\end{multicols}

\section{Conditionnelles}

\begin{definition}
Pour résoudre certains problèmes, il est parfois nécessaire de faire un test
pour savoir si on doit exécuter une tâche ou une autre. \emph{Si} le test est
vrai, \emph{alors} on exécute une tâche, \emph{sinon} on exécute une autre
tâche.
\end{definition}

\begin{multicols}{2}
\begin{exemple}~

  \begin{lstlisting}[language=naturel,frame=lines,mathescape=true]
    Afficher "Longueur des trois cotes."
    Lire A
    Lire B
    Lire C
    Si A = B et B = C
    Alors
      Afficher "Equilateral."
    Sinon
      Si A = B ou B = C ou A = C
      Alors
        Afficher "Isocele."
      Sinon
        Afficher "Scalene."
      FinSi
    FinSi
  \end{lstlisting}
\end{exemple}

\columnbreak

\begin{exercice}Que fait l'algorithme suivant ?
  \begin{lstlisting}[language=naturel,frame=lines,mathescape=true]
    Lire x
    Si x > 0
    Alors
      Afficher $\sqrt{\text{x}}$
    Sinon
      Afficher "Impossible"
    FinSi
  \end{lstlisting}

  Corriger l'algorithme pour qu'il affiche un résultat correct lorsque
  \texttt{x} vaut 0.
\end{exercice}

\begin{exercice}
  Écrire un algorithme qui prend en entrée les coordonnées d'un point, et qui
  affiche si ce point fait partie de la courbe représentative de la fonction
  $f:x\mapsto\dfrac{x-2}{x+1}$ ou non.
\end{exercice}
\end{multicols}

\section{Boucles}


\begin{multicols}{2}
Certains problèmes nécessitent de répeter un ensemble d'instructions plusieurs
fois.

Il existe deux types de boucles :
\begin{itemize}
  \item les boucles \texttt{Pour i allant de 1 à n} exécutent \texttt{n} fois la boucle, pour chacune des valeurs possibles de \texttt{i} ;
  \item les boucles \texttt{While condition} exécutent la boucle tant que la condition n'est pas remplie.
\end{itemize}

  \begin{exemple}Affichage de la table de multiplication de 9.
    \begin{lstlisting}[language=naturel,frame=lines,mathescape=true]
    Pour i allant de 1 a n
    Faire
      Afficher 9*i
    FinPour
    \end{lstlisting}
  \end{exemple}

  \columnbreak

  \begin{exemple}Calcul de l'entier $n$ tel que la somme des entiers de 1 à $n$
    fait 2016.
    \begin{lstlisting}[language=naturel,frame=lines,mathescape=true]
    0 -> somme
    0 -> n
    Tant que somme < 2016
    Faire
      n + 1 -> n
      somme + n -> n
    FinTantque
    Si somme = 2016
    Alors
      Afficher n
    Sinon
      Afficher "Impossible"
    FinSi
    \end{lstlisting}
  \end{exemple}
\end{multicols}

\section{Exercices}

\begin{exercice}
  Écrire un algorithme qui prend en argument les longueurs des trois côtés d'un
  triangle, et qui affiche si ce triangle est rectangle ou non.
\end{exercice}

\begin{exercice}
  Un magasin offre 5\% de réduction sur le montant total d'un achat si celui-ci
  est supérieur à 100~\euro. Écrire un algorithme qui lit en entrée le montant
  total, et affiche le montant après l'éventuelle réduction.
\end{exercice}

\begin{exercice}
  Une bactérie double sa population chaque jour. Écrire un algorithme qui
  calcule la population de bactéries après trente jours, à partir d'une seule
  bactérie au départ.
\end{exercice}

\begin{exercice}
  Un enfant veut acheter un jeu à 245 \euro. Ses parents lui donnent un euro la
  première semaine, deux euros la deuxième semaine, et ainsi de suite. Écrire
  un algorithme qui calcule à partir de combien de semaine l'enfant va-t-il
  pouvoir acheter son jeu ?
\end{exercice}

\begin{exercice}
  On appelle \emph{parfait} un nombre qui est égal à la somme de ses diviseurs,
  sauf lui même ($6 = 1 + 2 + 3$ est parfait ; $8 \neq 4 + 2 + 1$ n'est pas
  parfait). Écrire un algorithme qui vérifie si un nombre est parfait. Écrire
  un second algorithme qui cherche tous les nombres parfaits inférieurs à un
  nombre donné.
\end{exercice}
