\documentclass[12pt]{article}

\usepackage{pablo}
\usepackage[a4paper,margin=1cm]{geometry}
\usepackage{tabularx}
\usepackage{forloop}

\pagestyle{empty}

\usetikzlibrary{chains}

\newcounter{ct}
\newcommand{\probleme}[7]{
  \section[#1]{#1 \forloop{ct}{0}{\value{ct}<#2}{$\star$}}
  \begin{tabularx}{\textwidth}{Xc}
    \begin{description}
      \item[Nombre de colonnes] #5
      \item[Entrée] #3
      \item[Sortie] #4
    \end{description}
    &
    \tikzset{
      cube/.style={
        rectangle,
        draw=black,
        minimum width=1em,
      }
    }
    \begin{tabular}[t]{c||c}
      \\
      \begin{tikzpicture}[
          start chain=going above,
          node distance=1pt,
        ]
        \foreach \i in {1, ..., #5}
        {
          \draw[very thick] ({.7*\i},0) -- ++(.5, 0);
        }
        #6
      \end{tikzpicture}
      &
      \begin{tikzpicture}[
          start chain=going above,
          node distance=1pt,
        ]
        \foreach \i in {1, ..., #5}
        {
          \draw[very thick] ({.7*\i},0) -- ++(.5, 0);
        }
        #7
      \end{tikzpicture}
      \\
      Entrée & Sortie \\
      \multicolumn{2}{c}{\textbf{Exemple}}
    \end{tabular}
  \end{tabularx}
}
\begin{document}
\probleme{Inversion}{1}{
  Une pile de nombres sur la colonne de gauche
}{
  La même pile, éventuellement sur une autre colonne, inversée.
}{
  2
}{
  \node[anchor=south, on chain, cube] at (.95, 0) {1};
  \node[on chain, cube]{5};
  \node[on chain, cube]{7};
  \node[on chain, cube]{3};
}{
  \node[anchor=south, on chain, cube] at (1.65, 0) {3};
  \node[on chain, cube]{7};
  \node[on chain, cube]{5};
  \node[on chain, cube]{1};
}

\probleme{Décalage}{2}{
  Une pile de nombres sur la colonne de gauche
}{
  La même pile, décalée sur une colonne plus à droite
}{
  3
}{
  \node[anchor=south, on chain, cube] at (.95, 0) {1};
  \node[on chain, cube]{5};
  \node[on chain, cube]{7};
  \node[on chain, cube]{3};
}{
  \node[anchor=south, on chain, cube] at (2.35, 0) {1};
  \node[on chain, cube]{5};
  \node[on chain, cube]{7};
  \node[on chain, cube]{3};
}

\probleme{Parité}{1}{
  Une pile de nombres sur la colonne centrale
}{
  Les nombres pairs sur la colonne de gauche, les nombres impairs sur celle de droite. L'ordre des nombres n'a pas d'importance.
}{
  3
}{
  \node[anchor=south, on chain, cube] at (1.65, 0) {1};
  \node[on chain, cube]{5};
  \node[on chain, cube]{4};
  \node[on chain, cube]{7};
  \node[on chain, cube]{3};
  \node[on chain, cube]{2};
}{
  \node[anchor=south, on chain, cube] at (.95, 0) {2};
  \node[on chain, cube]{4};
  \node[anchor=south, on chain, cube] at (2.35, 0) {1};
  \node[on chain, cube]{5};
  \node[on chain, cube]{7};
  \node[on chain, cube]{3};
}

\probleme{Conjonction}{2}{
  Une pile de nombres sur la colonne de gauche
}{
  Les nombres bleus pairs sur la deuxième colonne, les nombres bleus impairs sur la troisième colonne, les cubes rouges sur la dernière colonne.
}{
  4
}{\scriptsize
  \node[anchor=south, on chain, cube] at (0.95, 0) {B1};
  \node[on chain, cube]{B5};
  \node[on chain, cube]{R4};
  \node[on chain, cube]{B7};
  \node[on chain, cube]{R3};
  \node[on chain, cube]{B2};
}{\scriptsize
  \node[anchor=south, on chain, cube] at (1.65, 0) {B2};
  \node[anchor=south, on chain, cube] at (2.35, 0) {B1};
  \node[on chain, cube]{B7};
  \node[on chain, cube]{B5};
  \node[anchor=south, on chain, cube] at (2.95, 0) {R4};
  \node[on chain, cube]{R3};
}

\probleme{Disjonction}{2}{
  Une pile de nombres sur la colonne de gauche
}{
  Les nombres bleus ou pairs sur la deuxième colonne, les autres sur la troisième colonne.
}{
  3
}{\scriptsize
  \node[anchor=south, on chain, cube] at (0.95, 0) {R1};
  \node[on chain, cube]{B5};
  \node[on chain, cube]{R4};
  \node[on chain, cube]{B7};
  \node[on chain, cube]{R3};
}{\scriptsize
  \node[anchor=south, on chain, cube] at (1.65, 0) {B7};
  \node[on chain, cube]{B5};
  \node[on chain, cube]{R4};
  \node[anchor=south, on chain, cube] at (2.35, 0) {R3};
  \node[on chain, cube]{R1};
}

\probleme{Parité 2}{2}{
  Deux piles de nombres sur les colonnes centrales
}{
  Les nombres pairs sur la colonne de gauche, les nombres impairs sur celle de droite. L'ordre des nombres n'a pas d'importance.
}{
  4
}{
  \node[anchor=south, on chain, cube] at (1.65, 0) {1};
  \node[on chain, cube]{7};
  \node[on chain, cube]{3};
  \node[on chain, cube]{2};
  \node[anchor=south, on chain, cube] at (2.35, 0) {4};
  \node[on chain, cube]{5};
}{
  \node[anchor=south, on chain, cube] at (.95, 0) {2};
  \node[on chain, cube]{4};
  \node[anchor=south, on chain, cube] at (3.05, 0) {1};
  \node[on chain, cube]{5};
  \node[on chain, cube]{7};
  \node[on chain, cube]{3};
}

\probleme{Maximum}{3}{
  Une pile de nombres sur la colonne centrale.
}{
  Le plus grand nombre au sommet de  la colonne de droite ; la position des autres nombres n'a pas d'importance.
}{
  3
}{
  \node[anchor=south, on chain, cube] at (1.65, 0) {1};
  \node[on chain, cube]{7};
  \node[on chain, cube]{3};
  \node[on chain, cube]{2};
  \node[on chain, cube]{5};
}{
  \node[anchor=south, on chain, cube] at (.95, 0) {2};
  \node[on chain, cube]{3};
  \node[anchor=south, on chain, cube] at (1.65, 0) {1};
  \node[anchor=south, on chain, cube] at (2.35, 0) {5};
  \node[on chain, cube]{7};
}

\probleme{Ordonner}{3}{
  Une pile de cubes rouges et bleus sur la colonne de gauche, mélangés.
}{
  Sur la colonne de droite, les cubes rouge en bas, et les bleus en haut.
}{
  3
}{
  \node[anchor=south, on chain, cube] at (0.95, 0) {B};
  \node[on chain, cube]{R};
  \node[on chain, cube]{R};
  \node[on chain, cube]{B};
  \node[on chain, cube]{R};
}{
  \node[anchor=south, on chain, cube] at (2.35, 0) {R};
  \node[on chain, cube]{R};
  \node[on chain, cube]{R};
  \node[on chain, cube]{B};
  \node[on chain, cube]{B};
}

\probleme{Échange}{2}{
  Une pile de cubes bleus à gauche, une pile de cubes rouges à droite.
}{
  Les mêmes piles, mais leur position est inversée.
}{
  3
}{
  \node[anchor=south, on chain, cube] at (2.35, 0) {R};
  \node[on chain, cube]{R};
  \node[on chain, cube]{R};
  \node[anchor=south, on chain, cube] at (0.95, 0) {B};
  \node[on chain, cube]{B};
}{
  \node[anchor=south, on chain, cube] at (0.95, 0) {R};
  \node[on chain, cube]{R};
  \node[on chain, cube]{R};
  \node[anchor=south, on chain, cube] at (2.35, 0) {B};
  \node[on chain, cube]{B};
}

\probleme{Ramassage}{3}{
  Quelques cubes dans chaque colonne.
}{
  Tous les cubes sur la colonne de gauche.
}{
  5
}{
  \node[anchor=south, on chain, cube] at (0.95, 0) {B};
  \node[on chain, cube]{B};
  \node[anchor=south, on chain, cube] at (1.65, 0) {B};
  \node[anchor=south, on chain, cube] at (2.35, 0) {B};
  \node[on chain, cube]{B};
  \node[on chain, cube]{B};
  \node[anchor=south, on chain, cube] at (3.75, 0) {B};
  \node[on chain, cube]{B};
}{
  \node[anchor=south, on chain, cube] at (0.95, 0) {B};
  \node[on chain, cube]{B};
  \node[on chain, cube]{B};
  \node[on chain, cube]{B};
  \node[on chain, cube]{B};
  \node[on chain, cube]{B};
  \node[on chain, cube]{B};
  \node[on chain, cube]{B};
}

\probleme{Distribution}{3}{
  Des cubes bleus sur la colonne de gauche.
}{
  Chaque colonne contient autant de cube, à un près, sauf la colonne de départ, qui est vide.
}{
  5
}{
  \node[anchor=south, on chain, cube] at (0.95, 0) {B};
  \node[on chain, cube]{B};
  \node[on chain, cube]{B};
  \node[on chain, cube]{B};
  \node[on chain, cube]{B};
  \node[on chain, cube]{B};
  \node[on chain, cube]{B};
}{
  \node[anchor=south, on chain, cube] at (1.65, 0) {B};
  \node[on chain, cube]{B};
  \node[anchor=south, on chain, cube] at (2.35, 0) {B};
  \node[on chain, cube]{B};
  \node[anchor=south, on chain, cube] at (3.05, 0) {B};
  \node[anchor=south, on chain, cube] at (3.75, 0) {B};
  \node[on chain, cube]{B};
}

\probleme{Croissant}{4}{
  Des cubes sur la première colonne.
}{
  Les cubes, triés par ordre croissant (le plus petit en dessous) sur la colonne de droite.
}{
  4
}{
  \node[anchor=south, on chain, cube] at (0.95, 0) {3};
  \node[on chain, cube]{5};
  \node[on chain, cube]{2};
  \node[on chain, cube]{1};
  \node[on chain, cube]{4};
}{
  \node[anchor=south, on chain, cube] at (3.05, 0) {1};
  \node[on chain, cube]{2};
  \node[on chain, cube]{3};
  \node[on chain, cube]{4};
  \node[on chain, cube]{5};
}

\probleme{Égalité}{2}{
  Un nombre pair de cubes sur la première colonne.
}{
  Autant de cubes sur la deuxième colonne que sur la troisième.
}{
  3
}{
  \node[anchor=south, on chain, cube] at (0.95, 0) {};
  \node[on chain, cube]{};
  \node[on chain, cube]{};
  \node[on chain, cube]{};
  \node[on chain, cube]{};
  \node[on chain, cube]{};
}{
  \node[anchor=south, on chain, cube] at (2.35, 0) {};
  \node[on chain, cube]{};
  \node[on chain, cube]{};
  \node[anchor=south, on chain, cube] at (1.65, 0) {};
  \node[on chain, cube]{};
  \node[on chain, cube]{};
}

\probleme{Quelle colonne ?}{2}{
  Entre 1 et 5 cubes sur la première colonne.
}{
  Tous les cubes placés sur la colonne correspondant à leur nombre, plus un (c'est-à-dire sur la seconde colonne s'il n'y a qu'un cube, sur la troisième colonne s'il y a deux cubes, etc.).
}{
  6
}{
  \node[anchor=south, on chain, cube] at (0.95, 0) {};
  \node[on chain, cube]{};
  \node[on chain, cube]{};
}{
  \node[anchor=south, on chain, cube] at (3.05, 0) {};
  \node[on chain, cube]{};
  \node[on chain, cube]{};
}

\end{document}
