\documentclass[11pt]{article}

\usepackage{pablo}
\usepackage{array}
\usepackage[a5paper,margin=0.5cm]{geometry}
\pagestyle{empty}

\begin{document}
\begin{center}
  Manipulation de la fonction trinôme
\end{center}

\noindent\emph{Un compte-rendu est attendu en fin de séance. Si nécessaire, vous pouvez déposer des fichiers numériques dans le dossier partagé prévu à cet effet.}

\begin{exercice}[Fonction trinôme]
Sur Geogebra, créer trois curseurs nommés $a$, $b$ et $c$ pouvant varier de $-5$ à $5$ selon des incréments de 0,1 pour $a$ et de $1$ pour $b$ et $c$ puis, dans la zone de saisie, créer la fonction $f(x)=a*x^2+b*x+c$.
\begin{enumerate}
 \item Donner à $a$ la valeur 0. Qu'observe-t-on ?\\
       Pour toute la suite on prendra $a\neq0$.
 \item Donner à $a$ la valeur 1 et à $b$ et $c$ la valeur 0.
       \begin{enumerate}
        \item De quelle nature est la courbe obtenue ?
        \item Indiquer l'abscisse de son sommet et ses éléments de symétrie.
        \item Donner l'expression de $f(x)$.
        \item Par lecture graphique, dresser le tableau des variations de $f$.
       \end{enumerate}
 \item Donner à $b$ et $c$ la valeur 0 et faire varier $a$.
       \begin{enumerate}
        \item Quel semble être le \og rôle \fg{} de $a$ ?
	\item Dans quel cas le tableau de variations de $f$ est-il identique au précédent et dans quel cas est-il différent ?
       \end{enumerate}
 \item Donner à $a$ la valeur 1, à $b$ la valeur 0 et faire varier $c$.
       \begin{enumerate}
        \item Quel semble être le \og rôle \fg{} de $c$ ?
        \item Que peut-on dire de l'intersection de la courbe avec l'axe des ordonnées ?
        \item Démontrer par le calcul que toute fonction de la forme $f(x)=ax^2+bx+c$ coupe l'axe des ordonnées en un point dont les coordonnées ne dépendent que de $c$.
       \end{enumerate}
 \item	On notera $x_0$ l'abscisse du sommet de la courbe.
	\begin{enumerate}
        \item Donner à $a$ la valeur 1, à $c$ la valeur 0 et faire varier $b$.\\
	Compléter le tableau suivant :
	\begin{center}
	  \begin{tabular}{m{0.7cm}*{11}{|m{0.5cm}}}
	  \centering $b$ & $-5$ & $-4$ & $-3$ & $-2$ & $-1$ & 0 & 1 & 2 & 3 & 4 & 5 \\ \hline
	  \centering $x_0$ &&&&&&&&&&& \\
	  \end{tabular}
	\end{center}
	\item Donner à $a$ la valeur 2, à $c$ la valeur 0 et faire varier $b$.\\
	Compléter le tableau suivant :
	\begin{center}
	  \begin{tabular}{m{0.7cm}*{11}{|m{0.5cm}}}
	  \centering $b$ & $-5$ & $-4$ & $-3$ & $-2$ & $-1$ & 0 & 1 & 2 & 3 & 4 & 5 \\ \hline
	  \centering $x_0$ &&&&&&&&&&& \\
	  \end{tabular}
	\end{center}
	\item Donner à $a$ la valeur $-0,5$, à $c$ la valeur 0 et faire varier $b$.\\
	Compléter le tableau suivant :
	\begin{center}
	  \begin{tabular}{m{0.7cm}*{11}{|m{0.5cm}}}
	  \centering $b$ & $-5$ & $-4$ & $-3$ & $-2$ & $-1$ & 0 & 1 & 2 & 3 & 4 & 5 \\ \hline
	  \centering $x_0$ &&&&&&&&&&& \\
	  \end{tabular}
	\end{center}
	\item Faire varier $c$. Cela influence-t-il $x_0$ ?
	\item Conjecturer l'expression de $x_0$ en fonction de $a$ et $b$.\\
	      Que peut-on dire des éléments de symétrie de la courbe dans tous les cas ?
       \end{enumerate}
       \item Régler le curseur $b$ pour que son incrément soit maintenant de 0,1.\\
	     On admettra qu'un projectile lancé en l'air suit une trajectoire parfaitement parabolique.\\
	     Un projectile est lancé depuis une colline depuis une altitude de 400\,m symbolisée par le point $A\,(0\,;\,4)$. Il doit atteindre une cible située à 1\,000\,m à l'altitude 0, symbolisée par le point $B\,(10\,;\,0)$. Pour des raisons de sécurité, son altitude maximum ne doit pas dépasser 800\,m.\\
	     Déterminer des valeurs de $a$, $b$ et $c$ permettant d'obtenir une courbe symbolisant la trajectoire de ce projectile et satisfaisant toutes ces conditions.
\end{enumerate}
\end{exercice}

\begin{exercice}[Forme canonique]
\emph{Cette activité nécessite l'utilisation du logiciel {Geogebra}.}\\
Sur Geogebra, créer trois curseurs nommés $\alpha$, $\beta$ et $\gamma$ pouvant varier de $-5$ à $5$ selon des incréments de 0,5  puis, dans la zone de saisie, créer la fonction $f(x)=\alpha*(x-\beta)^2+\gamma$.
\begin{enumerate}
 \item \begin{enumerate}
        \item Dans la zone de saisie, créer la fonction $g(x)=2x^2-2x+4$.
	\item Déterminer les valeurs de $\alpha$, $\beta$ et $\gamma$ telles que la courbe de $f$ et celle de $g$ soient confondues.
	\item Vérifier par le calcul que les deux fonctions sont bien égales.
	\item Noter l'abscisse du sommet de la courbe.
	\item Par le calcul, déterminer l'ensemble des solutions de l'équation $g(x)=0$.\\ Comment cela se traduit-il graphiquement ?
       \end{enumerate}
 \item Mêmes questions avec $g(x)=-1,5x^2-6x-4,5$.
 \item Mêmes questions avec $g(x)=-0,5x^2-2x-1,5$.
 \item \begin{enumerate}
        \item Conjecturer quelles doivent être les valeurs de $\alpha$ et de $\beta$.
	\item \textbf{Par le calcul}, en utilisant la conjecture précédente, déterminer les valeurs de $\alpha$, $\beta$ et $\gamma$ pour que la fonction $f$ soit égale à la fonction $g(x)=2x^2-4x-1$.
	\item Déduire les valeurs exactes des coordonnées des points d'intersection de la courbe de $g$ avec l'axe des abscisses. \\ Vérifier si vos résultats co\"incident avec la courbe de la fonction sur Geogebra.
       \end{enumerate}
\end{enumerate}
\end{exercice}

\end{document}
