\chapter{Échantillonnage}
\begin{activite}
  Je lance 30 fois une pièce de monnaie, et j'obtiens 8 fois Pile et 22 fois Face. Je me demande si cette pièce est équilibrée ou non.

  Proposer une réponse, ou une méthode pour répondre à cette question.
\end{activite}

\section{Échantillon}

\begin{definition}
  On appelle \emph{échantillon de taille $n$} les résultats de $n$ répétitions indépendantes d'une même expérience aléatoire.
\end{definition}

\begin{exemple}~
  \begin{itemize}
    \item Pour évaluer la qualité des produits à la sortie d'une usine, on en prélève 100 au hasard, et pour chacun d'entre eux, on détermine s'il est considéré défectueux ou non.
    \item Pour tenter de deviner les résultats d'une élection, on prélève au hasard 1000 personnes dans la population, et on leur demande quelle est leur intention de vote.
  \end{itemize}
\end{exemple}

\section{Intervalle de fluctuation}

\begin{activite}
  J'affirme que 95 fois sur 100, si on lance 30 fois de suite une pièce de monnaie équilibrée, on obtient entre 10 et 20 fois Pile (inclus).

  Étant donné cette nouvelle information, que peut-on dire d'une pièce qui a donné 8 pile en 30 lancers ?
\end{activite}

\begin{definition}
  L'\emph{intervalle de fluctuation} au seuil de 95\%, relatif aux
  échantillons de taille $n$, est l’intervalle centré autour de $p$,
  proportion du caractère dans la population, où se situe, avec une
  probabilité égale à $0,95$, la fréquence observée.
\end{definition}

\begin{propriete}
  Pour un échantillon de taille $n\geq25$, et une proportion $p$ du
  caractère appartenant à $[0,2;0,8]$, la fréquence observée d'apparition du caractère dans l'échantillon appartient à l'intervalle $\left[p-\frac{1}{\sqrt{n}};p+\frac{1}{\sqrt{n}}\right]$ avec une probabilité d'au moins $0,95$.
\end{propriete}

\begin{exemple}[Estimation d'une proportion inconnue]
  TODO
\end{exemple}

\begin{exemple}[Prise de décision]
  TODO
\end{exemple}
