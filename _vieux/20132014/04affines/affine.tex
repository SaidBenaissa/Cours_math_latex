\documentclass[11pt]{article}

\usepackage{pablo}
\usepackage[a5paper,margin=1.9cm]{geometry}

\pagestyle{empty}

\begin{document}

\begin{center}\large
  \textsc{1 --- Fonctions affines et linéaires}
\end{center}

\begin{definition}
  Une \emph{fonction affine} est une fonction qui s'écrit sous la forme \blanc{$x\mapsto
  ax+b$} où $a$ et $b$ sont réels. Elle est définie sur \blanc{$\mathbb R$}.

  Quand $b=0$, la fonction est de la forme $x\mapsto ax$, et on dit alors que
  la fonction est \blanc{\emph{linéaire}}.
\end{definition}

\begin{propriete}
  Soit $f:x\mapsto ax+b$ une fonction affine.
  \begin{itemize}
    \item si $a>0$, la fonction est \blanc{croissante} sur $\mathbb R$;
    \item si $a=0$, la fonction est \blanc{constante} sur $\mathbb R$;
    \item si $a<0$, la fonction est \blanc{décroissante} sur $\mathbb R$.
  \end{itemize}
\end{propriete}

\subsection*{Représentation graphique}

\begin{propriete}D'une manière générale, la courbe représentative d'une
  fonction affine est une \blanc{droite}.

  \begin{itemize}
    \item Si la fonction est linéaire, cette droite passe par \blanc{l'origine du repère}.
    \item Si la fonction est constante, cette droite est \blanc{parallèle à l'axe des abscisses}.
    \end{itemize}
\end{propriete}

\begin{definition}Soit une fonction affine $f:x\mapsto ax+b$, et $\cal D$ sa courbe représentative.
  \begin{itemize}
    \item Le réel $a$ est appelé \blanc{\emph{coefficient directeur}}.
    \item Le réel $b$ est appelé \blanc{\emph{ordonné à l'origine}}.
  \end{itemize}
\end{definition}

\begin{propriete}
  Soient $f:x\mapsto ax+b$ une fonction affine, $\cal D$ sa courbe
  représentative, et $A(x_A,y_A)$ et $B(x_B,y_B)$ deux points de $D$. Alors
  \blanc{$a=\frac{y_B-y_A}{x_B-x_A}$}.
\end{propriete}

\begin{methode}[Détermination de l'équation d'une fonction affine] Soit $f$ une fonction affine dont on connaît la représentation graphique $\cal D$.

  Pour calculer $a$, on choisit deux points arbitraires $A$ et $B$ de $\cal
  D$, et on calcule $a=\frac{y_B-y_A}{x_B-x_A}$.

  Pour calculer $b$, on lit l'ordonnée à l'origine, c'est-à-dire l'ordonnée
  du point d'intersection de $\cal D$ avec l'axe des ordonnées.
\end{methode}

\end{document}
