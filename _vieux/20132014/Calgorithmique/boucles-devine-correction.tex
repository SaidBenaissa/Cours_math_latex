\documentclass[12pt]{article}

\usepackage{pablo}

\usepackage[a4paper,margin=1.0cm]{geometry}
\usepackage{savetrees}
\usepackage{multicol}
\usepackage{url}

\usepackage{pablo-listings}

\pagestyle{empty}

\begin{document}

\begin{center}
  Algorithmique

  {\large
    \textsc{Utilisation des boucles}
  }

  ------------------
\end{center}

\section*{Deviner un nombre}

Écrire un programme qui :
\begin{itemize}
  \item choisit un nombre au hasard entre 0 et 100 ;
  \item tant que l'utilisateur n'a pas deviné ce nombre, lui demande un nombre, et affiche \texttt{Trop haut} ou \texttt{Trop bas} suivant que sa proposition est supérieure ou inférieure au nombre recherché ;
  \item affiche \texttt{Bravo} avant de terminer.
\end{itemize}

\section*{Algorithmes}

Voici quatre versions de l'algorithme demandé. La première (langue naturelle)
est une description en langue française. Les deux suivantes sont les programmes
correspondants, pour calculatrices Casio ou TI. La dernière est le programme
correspondant écrit en Python3 (logiciel installé sur les ordinateurs de
l'école, téléchargeable librement sur \url{http://www.python.org}).

Tout comme il y a plusieurs manières de résoudre un même problème mathématique,
il y a plusieurs algorithmes résolvant le problème, et plusieurs programmes
mettant en œuvre cet algorithme. Les solutions données ici ne sont qu'une
possibilité parmi d'autres.

\begin{multicols}{2}
  \begin{minipage}{\textwidth}
\subsection*{Langue naturelle}

Choisir un nombre $A$ aléatoirement entre 0 et 100.\\
Initialiser $X$ à $-1$.\\
Tant que $A$ est différent de $X$, répéter :\\
Demander un nombre $X$.\\
Si $X$ est plus petit que $A$, afficher \verb+Trop petit+.\\
Si $X$ est plus grand que $A$, afficher \verb+Trop grand+.\\
Fin de la boucle.\\
Afficher \verb+Bravo+.
\end{minipage}

\columnbreak

  \section*{Calculatrice TI}
\begin{lstlisting}[language=TI,frame=single]
ent(100 * NbrAleat) -> A
-1 -> X
While A neq X
Prompt X
If X < A
Then
Disp "Trop petit"
End
If X > A
Then
Disp "Trop grand"
End
End
Disp "Bravo"
\end{lstlisting}
\end{multicols}

\begin{multicols}{2}
\section*{Calculatrice Casio}
\begin{lstlisting}[language=casio,frame=single]
Int(100 * Ran#) -> A
-1 -> X
While A neq X
? -> X
If X < A
Then
"Trop petit"
IfEnd
If X > A
Then
"Trop grand"
IfEnd
WhileEnd
"Bravo"
\end{lstlisting}

\columnbreak

\section*{Python3}
\begin{lstlisting}[language=python,frame=single]
import random
A = random.randint(1, 100)
X = -1
while A != X:
    X = int(input("? "))
    if X < A:
        print("Trop petit")
    if X > A:
        print("Trop grand")
print("Bravo")
\end{lstlisting}
\end{multicols}

\end{document}
