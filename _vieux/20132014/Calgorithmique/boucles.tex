\documentclass[12pt]{article}

\usepackage{pablo}

\usepackage[a5paper,margin=1.0cm]{geometry}
\usepackage{savetrees}

\pagestyle{empty}

\begin{document}

\begin{center}
  Algorithmique

  {\large
    \textsc{Utilisation des boucles}
  }

  ------------------
\end{center}

\section*{Suite de Syracuse}

Étant donné un nombre $A$, on calcule une suite de nombres (appelée
\emph{suite de Syracuse}) de la manière suivante :
\begin{itemize}
  \item si $A$ est pair, on le divise par 2 ;
  \item si $A$ est impair, on le multiplie par 3, puis on ajoute 1.
\end{itemize}
Puis on recommence avec le nouveau nombre obtenu.

\begin{enumerate}
  \item Calculer les dix premiers nombres de cette suite, avec 17 comme
    nombre initial.
  \item Écrire un programme \texttt{SYRACUSE} sur la calculatrice qui :
    \begin{itemize}
      \item demande à l'utilisateur deux nombres $A$ et $N$ ;
      \item calcule les $N$ premiers éléments de la suite de Syracuse avec
        $A$ comme nombre initial.
    \end{itemize}
    \item Faire plusieurs essais, et énoncer une conjecture concernant le
      comportement de la suite.
    \item Prouver cette conjecture.
\end{enumerate}

\section*{Deviner un nombre}

Écrire un programme qui :
\begin{itemize}
  \item choisit un nombre au hasard entre 1 et 100 ;
  \item tant que l'utilisateur n'a pas deviné ce nombre, lui demande un nombre, et affiche \texttt{Trop haut} ou \texttt{Trop bas} suivant que sa proposition est supérieure ou inférieure au nombre recherché ;
  \item affiche \texttt{Bravo} avant de terminer.
\end{itemize}

\section*{Calcul du PGCD}

\begin{enumerate}
  \item Rappeler la définition du \textsc{pgcd}.
  \item Rappeler, au choix, l'algorithme d'Euclide ou l'algorithme des
    différences, utilisé pour calculer le \textsc{pgcd} de deux nombres.
  \item Quel est le \textsc{pgcd} de 15 et 20 ? De 111 et 51 ?
  \item Écrire un programme \texttt{PGCD} sur la calculatrice qui :
    \begin{itemize}
      \item demande à l'utilisateur deux nombres $A$ et $B$ ;
      \item calcule le \textsc{pgcd} de $A$ et $B$ ;
      \item affiche ce \textsc{pgcd}.
    \end{itemize}
  \item Exécuter ce programme pour calculer le \textsc{pgcd} de 15 et 20 ; de 111 et
    51 ; de 1729 et 521.
\end{enumerate}

\end{document}
