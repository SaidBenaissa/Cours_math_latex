\chapter{Statistiques descriptives}
\section{Vocabulaire}

TODO

\begin{itemize}
  \item Population, individu
  \item Caractère, qualitatif ou quantitatif, discret ou continu
  \item classe, intervalle, étendue
  \item Effectif total, d'une valeur
  \item Fréquence, effectifs cumulés croissants,fréquences cumulées croissantes
\end{itemize}

\section{Représentations graphiques}

\subsection{Histogramme}

TODO Description, exemple.

\subsection{Nuage de points}

TODO Description, exemple.

\subsection{Courbe des effectifs cumulés}

TODO Description, exemple.

\subsection{Diagramme circulaire}

TODO Description, exemple.


\section{Indicateurs d'une série statistique}

\subsection{Définitions}

\begin{definition}[Moyenne]
  La \emph{moyenne} d'une série de caractères $x_1, x_2, \ldots x_k$ et d'effectifs $n_1, n_2 \ldots n_k$ est la valeur notée $\bar x$ valant :
  \[{\bar x}=\dfrac{n_1x_1+n_2x_2+\ldots+n_kx_k}{n_1+n_2+\ldots+n_k}.\]
\end{definition}

\begin{definition}[Médiane]
  La \emph{médiane} d'une série est une valeur $m$ telle que la moitié des effectifs étudiés ait une valeur inférieure ou égale à $m$, et l'autre moité ait une valeur supérieure ou égale.
\end{definition}

\begin{remarque}
  La moyenne et la médiane sont des indicateurs de \emph{position}.
\end{remarque}

\begin{definition}[Quartiles]~
  \begin{itemize}
    \item Le \emph{premier quartile} d'une série est la plus petite valeur telle qu'au moins $25\%$ des valeurs lui soit inférieure ou égale.
    \item Le \emph{troisième quartile} d'une série est la plus petite valeur telle qu'au moins $75\%$ des valeurs lui soit inférieure ou égale.
  \end{itemize}
\end{definition}

\begin{remarque}
  Les quartiles sont des indicateurs de \emph{dispersion}
\end{remarque}

\subsection{Méthodes}
\subsubsection{Calcul de la médiane}
On commence par trier les caractères par ordre croissant.
\begin{itemize}
  \item Si la série étudiée a un effectif $N$ impair, la médiane est la valeur du caractère numéroté $\frac{N+1}{2}$.
  \item Si la série étudiée a un effectif $N$ pair, la médiane est la moyenne des caractères numérotés $\frac{N}{2}$ et $\frac{N}{2}+1$.
\end{itemize}

\subsubsection{Calcul des quartiles}

On trie les effectifs par ordre croissant, et on applique la définition.

\begin{exemple}
Calculer la médiane et les quartiles des salaires des entreprises A et B.

  A :
  \begin{tabular}{r|c|c|c|c|c|c|c}
    Salaire  & 1450 & 1500 & 1550 & 1600 & 1650 & 1700 & 1750 \\
    \hline
    Effectif & 5    &  2   &  5   &  7   &  3   &  6   & 7    \\
  \end{tabular}

  B :
  \begin{tabular}{r|c|c|c|c|c|c|c}
    Salaire  & 1450 & 1460 & 1470 & 1580 & 1600 & 2000 & 5000 \\
    \hline
    Effectif & 8    &  2   & 15   &  2   &  5   &  2   & 1    \\
  \end{tabular}
\end{exemple}


\section{Utilisation de la calculatrice}

TODO
