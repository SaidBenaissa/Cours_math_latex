\documentclass{article}

\usepackage{pablo}
\usepackage[a5paper,margin=1cm]{geometry}
\usepackage{savetrees}



\begin{document}

\thispagestyle{empty}

\setcounter{nth-gaps}{1}
\setcounter{section}{2}

\section{Indicateurs d'une série statistique}

\begin{definition}[Moyenne]
  La moyenne d'une série de caractères $x_1, x_2, \ldots x_k$ et d'effectifs $n_1, n_2 \ldots n_k$ est la valeur notée $\bar x$ valant :
  %\[{\bar x}=\dfrac{n_1x_1+n_2x_2+\ldots+n_kx_k}{n_1+n_2+\ldots+n_k}.\]

  \blanc{formule de la moyenne}
\end{definition}

\begin{definition}[Médiane]
  La médiane d'une série est une valeur $m$ telle que la moitié des effectifs étudiés ait une valeur inférieure ou égale à $m$, et l'autre moité ait une valeur supérieure ou égale.
\end{definition}

\begin{remarque}
  La moyenne et la médiane sont des indicateurs de \blanc{position}.
\end{remarque}

\begin{definition}[Quartiles]~
  \begin{itemize}
    \item Le premier quartile d'une série est la plus petite valeur telle qu'au moins $25\%$ des valeurs lui soit inférieure ou égale.
    \item Le troisième quartile d'une série est la plus petite valeur telle qu'au moins $75\%$ des valeurs lui soit inférieure ou égale.
  \end{itemize}
\end{definition}

\begin{remarque}
  Les quartiles sont des indicateurs de \blanc{dispersion}
\end{remarque}

\subsection*{Méthode}
\subsubsection*{Calcul de la médiane}
On commence par trier les caractères par ordre croissant.
\begin{itemize}
  \item Si la série étudiée a un effectif $N$ impair, la médiane est la valeur du caractère numéroté $\frac{N+1}{2}$.
  \item Si la série étudiée a un effectif $N$ pair, la médiane est la moyenne des caractères numérotés $\frac{N}{2}$ et $\frac{N}{2}+1$.
\end{itemize}

\subsubsection*{Calcul des quartiles}

On trie les effectifs par ordre croissant, et on applique la définition.

\subsection*{Exemple}

Calculer la médiane et les quartiles des salaires des entreprises A et B.

  A :
  \begin{tabular}{r|c|c|c|c|c|c|c}
    Salaire  & 1450 & 1500 & 1550 & 1600 & 1650 & 1700 & 1750 \\
    \hline
    Effectif & 5    &  2   &  5   &  7   &  3   &  6   & 7    \\
  \end{tabular}

  \vspace{\stretch{1}}

  B :
  \begin{tabular}{r|c|c|c|c|c|c|c}
    Salaire  & 1450 & 1460 & 1470 & 1580 & 1600 & 2000 & 5000 \\
    \hline
    Effectif & 8    &  2   & 15   &  2   &  5   &  2   & 1    \\
  \end{tabular}

  \vspace{\stretch{1}}

\end{document}
