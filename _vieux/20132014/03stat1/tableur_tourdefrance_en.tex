\documentclass[11pt]{article}

\PassOptionsToPackage{english}{babel}
\usepackage{pablo}
\selectlanguage{english}
\usepackage{eurosym}
\usepackage[a5paper,margin=1.5cm]{geometry}

\pagestyle{empty}
\begin{document}

\begin{center}
  {\large
    Spreadsheet
    ---
    \textsc{Tour de France}
  }
\end{center}

\noindent Open file : \texttt{tour\_de\_france\_2009.ods}.

\noindent The spread sheet contains the winners of \emph{Tour de France} from 1903 to 2009, the distance that was run (km), their average speed (km/h), and some other information.



\begin{question}[Using a spreadsheet to automate calculations]~

  \begin{enumerate}
      \setcounter{enumi}{-1}
    \item Give the formula linking time and distance to speed.
    \item What is the formula one have to write in cell $I3$ to calculate the race time in 1990 ?
      Copy this formula down to $I98$, to get the race time of every Tour.
    \item Choose the appropriate number format for column $I$, so that numbers are displayed using two decimals.
    \item In cell $F99$, calculate the mean of cells from $F3$ to $F98$.
    \item We call ``cumulative average speed since 1903'' the average speed an imaginary rider would have had if he had won every Tour since 1903.

      For instance, to calculate the cumulative average speed in 2000, we divide the total distance of Tours since 1903 by the sum of the winners' race times.

      Calculate, in column $J$, the cumulative average speeds since 1903.

    \item Explain why results displayed in $F99$ and $J99$ are different.

  \end{enumerate}
\end{question}


\begin{question}[Statistics]

  In this part, the array we are studying is the list of average speeds (column $F$ of the spreadsheet).

  \begin{enumerate}
    \item Copy values from $F3$ to $F98$ to column $L$ (from $L3$ to $L98$), and sort this latter column $L$.

    \item Fill column $M$ (ranks) with integers from 1 to 96, using a formula that will written once, then copied.

    \item Fill cells $L102$ to $L106$, using formulas.

    \item By analysing columns $L$ and $M$, find the minimum and maximum, the first and third quartiles, the median, and put these values in cells $M102$ to $M106$.

  \end{enumerate}
\end{question}

\newpage

\begin{center}
  {\large
    Spreadsheet
    ---
    \textsc{Optimisation}
  }
\end{center}

\noindent Mrs Cocoa is the director of a supermarket. She buys chocolate boxes from a factory, at 5 \officialeuro{} per box.

\noindent She sells these boxes in her store, at 13.60 \officialeuro{} each.
She usually sells 3000 boxes a week.

\noindent Mrs Cocoa orders a market study that shows that for each 10 cents discount, 100 more boxes a week are sold.

\noindent We are going to help Mrs Cocoa to choose the retail price of one chocolate box, in order to maximize her benefit.

Copy and complete the following array (all prices are in euros) on a spreadsheet, and decide on the discount Mrs Cocoa should offer to maximize her benefit.

~

{\small
\noindent
\hspace{-2em}
\begin{tabular}{|l|l|l|l|l|l|}
  \hline
  Number of
  & Retail price
  & Number of
  & 
  & Total
  & 
  \\
  discounts
  & of one box
  & boxes sold
  & Receipt
  & factory price
  & Benefit
  \\
  \hline
  0 & 13.60 & 3000 & & & \\
  1 & 13.50 & 3100 & & & \\
  \ldots & \ldots & \ldots & & & \\
  \hline
\end{tabular}
}

\end{document}
