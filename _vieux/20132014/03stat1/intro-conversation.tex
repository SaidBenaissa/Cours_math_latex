\documentclass[12pt]{article}

%\usepackage{savetrees}

% accents
\usepackage[utf8x]{inputenc}
\usepackage[francais]{babel}

\usepackage[a5paper,margin=1cm]{geometry}

\pagestyle{empty}

\begin{document}

\begin{center}
\textbf{
\large
  La répartition des tâches entre les femmes et les hommes dans le travail de la conversation
}

\textsc{
  Corinne Monnet
}
\end{center}

\noindent \fbox{\parbox{\textwidth}{
{L'auteure commente une expérience de West et Zimmerman, publiée dans \emph{Henley, N. (1975). "Power, Sex, and nonverbal Communication", in Henley \& Thorne (eds). Language and Sex : Difference and Dominance. Rowley, MA : Newbury House, 1975}.}
}}

\vspace{1cm}

J’en viens maintenant à l’étude proprement dite portant sur des dialogues enregistrés dans des lieux publics d’une communauté universitaire. Nous avons 20 couples non mixtes (10 couples femme/femme et 10 couples homme/homme) et 11 couples mixtes (composés exclusivement d’étudiant-e-s à une exception près où la femme est assistante). Les sujets de conversation varient depuis les échanges de politesse jusqu’à des sujets plus intimes, selon que ces personnes se rencontrent pour la première fois ou bien se connaissent davantage. Alors qu’elles dénombrent 22 chevauchements et 7 interruptions dans les dialogues non mixtes, elles trouvent 9 chevauchements et 48 interruptions dans les dialogues mixtes. On peut faire plusieurs remarques sur ces résultats.

Les chevauchements sont plus fréquents que les interruptions dans les dialogues non mixtes que dans les dialogues mixtes. Par contre, les interruptions sont beaucoup plus fréquentes en mixité que les chevauchements. Seuls 3 des 10 dialogues non mixtes comportent des interruptions, qui sont de plus réparties assez symétriquement entre les interlocutrices/teurs, alors que seul 1 dialogue mixte sur les 11 en est indemne. Les interruptions apparaissent donc comme systématiques dans les dialogues mixtes.

La plupart des chevauchements et interruptions sont dus aux hommes. \emph{Dans 96\% des cas, ce sont les hommes qui interrompent les femmes.} Nous sommes bien loin d’une distribution aléatoire des interruptions et le moins que l’on puisse dire, c’est qu’il y a une forte dominance masculine quant aux interruptions dans les dialogues femme/homme. Après avoir refait une étude dans des conditions différentes portant sur cinq conversations mixtes avec des personnes qui ne se connaissaient pas du tout, West et Zimmerman retrouvent toujours, à peu de chose près, les mêmes résultats.

\end{document}
