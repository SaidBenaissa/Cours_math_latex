\documentclass[12pt]{article}

\usepackage{pablo}
\usepackage[a5paper,margin=1.1cm]{geometry}
\pagestyle{empty}

\renewcommand{\thesubsubsection}{\arabic{subsubsection}}

\begin{document}
\begin{center}
  \textsc{Méthodologie}

{\Large
  Résolution de systèmes 
  de deux équations
  
  linéaires
à deux inconnues
}
\end{center}

\setcounter{subsubsection}{-1}
\subsubsection{Par résolution graphique}
\begin{compactenum}
\item Tracer les droites correspondant aux deux équations.
\item Lire les solutions sur le graphique (ce sont les points d'intersection des droites).
\item Dans le cas d'une solution unique, vérifier si c'est une solution exacte
  en remplaçant dans le système les inconnues par les valeurs trouvées.
\end{compactenum}

\subsubsection{Par substitution}
\begin{compactenum}
\item Isoler une inconnue (par exemple $x=\ldots$) dans une des équations.
\item Remplacer dans la deuxième équation l'inconnue $x$ par l'expression trouvée.
\item Résoudre cette deuxième équation pour trouver la valeur de $y$.
\item Utiliser cette valeur dans la première équation pour trouver $x$.
\end{compactenum}

\subsubsection{Par combinaison linéaire}
Nous nommons les équations $L_1$ et $L_2$.

Les opérations suivantes ne modifient pas les solutions d'un système.
  \begin{compactitem}
  \item $L_1 \leftrightarrow L_2$ : Permuter deux équations.
  \item $L_1 \rightarrow \alpha \times L_1$ : Multiplier les deux membres d'une équation par le même nombre non nul.
  \item $L_1 \rightarrow L_1 + L_2$ : Remplacer une équation par la somme de cette même équation avec une autre.
  \item \textit{Substitution} : Substituer dans une équation une inconnue par son expression exprimée dans une autre équation.
  \end{compactitem}
  La méthode consiste, en utilisant les opérations citées ci-dessus, à \og{}faire disparaitre\fg{} dans une équation toutes les inconnues sauf une, pour trouver sa valeur, et calculer la seconde inconnue par substitution.
\end{document}
