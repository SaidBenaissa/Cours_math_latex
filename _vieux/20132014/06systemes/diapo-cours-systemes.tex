\documentclass[15pt,smaller,aspectratio=169,xcolor={usenames,dvipsnames}]{beamer}
\usepackage{tikz}
\usepackage[utf8]{inputenc}
\usepackage[T1]{fontenc}
\usepackage[frenchb]{babel}
\usepackage{lmodern}
\usenavigationsymbolstemplate{}
\useinnertheme[shadow=true]{rounded}
\useoutertheme{infolines}
\usecolortheme{crane}

\setbeamerfont{block title}{size={}}
\setbeamercolor{titlelike}{parent=structure,bg=yellow!85!orange}

\newtheorem*{propriete}{Propriété}
\newtheorem*{corollaire}{Corollaire}
\newcommand{\systeme}[2]{\left\{\begin{array}{l}#1\\#2\end{array}\right.}

\title{Systèmes et droites}
\date{}

\begin{document}


\section{Systèmes d'équations linéaires}

\begin{frame}[label=pagesimple]


  \begin{block}{3 --- Systèmes d'équations linéaires}
  \end{block}

\pause

\begin{propriete}[Interprétation géométrique]
  Soit $(S)$ le système d'équations $\systeme{ax+by=c}{a'x+b'y=c'}$, et $d$ et $d'$ les droites définies par chacune des deux équations de $(S)$. Les solutions de $(S)$ sont les coordonnées des points d'intersection de $d$ et $d'$.
\end{propriete}

\pause

\begin{corollaire}Avec le même système $(S)$, trois cas seulement sont possibles :
  \begin{itemize}
    \item Le systeme $(S)$ a une infinité de solutions si et seulement si les droites $d$ et $d'$ sont confondues.
    \item Le système $(S)$ a une unique solution si et seulement si les droites $d$ et $d'$ sont sécantes.
    \item Le système $(S)$ n'a pas de solutions si et seulement si les droites $d$ et $d'$ sont parallèles et non confondues.
  \end{itemize}
\end{corollaire}
\end{frame}



\end{document}
