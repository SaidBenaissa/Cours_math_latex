% Title: Demonstration sans parole : repere cartesien
% Author: Jacques Duma
% Site: http://math.et.info.free.fr/TikZ/index.html

\documentclass{article}

\usepackage[screen,margin=0cm]{geometry}
\usepackage{calc}
\usepackage{tikz}
 \usetikzlibrary{calc}
\usepackage{pablo}

\begin{document} 

\pagestyle{empty}

% compteur auxiliaire pour le nombre de lignes 
\newcounter{GrilleTriangulaire}
% commande de trac� de la grille en environnement tikzpicture
% \GrilleTriangulaire{<largeur>}{<hauteur>)}
% <largeur> nombre d'hexagones en largeur
% <hauteur> nombre d'hexagones en hauteur
\newcommand{\GrilleTriangulaire}[2]{
\clip ({-#1*sqrt(3)/2+1},-#2) rectangle ({sqrt(3)/2*#1},#2);
\setcounter{GrilleTriangulaire}{\maxof{#1}{#2+#1/2}}
% 3 balayages par des droites paralleles 
\foreach \a in {-\theGrilleTriangulaire,...,\theGrilleTriangulaire}
  \draw[dotted,shift={(-60:\a)}]
    (-120:-\theGrilleTriangulaire)--(-120:\theGrilleTriangulaire);
\foreach \a in {-\theGrilleTriangulaire,...,\theGrilleTriangulaire}
  \draw[dotted,shift={(60:\a)}]
    (120:-\theGrilleTriangulaire)--(120:\theGrilleTriangulaire);
\foreach \a in {-\theGrilleTriangulaire,...,\theGrilleTriangulaire}
\draw[dotted,shift={(0,{sqrt(3)*\a/2})}]
    (-\theGrilleTriangulaire,0)--(\theGrilleTriangulaire,0);
  }
    
\vfill

\noindent\begin{tikzpicture}
	\GrilleTriangulaire{14}{9}

  \Large
  \coordinate (i) at (1,0);
  \coordinate (j) at (60:2);
  \draw[thick,->,blue] (0,0)  node {$\times$} node[below left]{$O$};
  \draw[thick,->,blue] (0,0) -- (i) node[midway,below]{$\vecteur\imath$};
  \draw[thick,->,blue] (0,0) -- (j) node[midway,left]{$\vecteur\jmath$};
  \draw[thick,blue] ($2*(i)-1/2*(j)$) node{$\times$} node[below left]{$A$};
  \draw[thick,blue] ($-2*(i)$) node{$\times$} node[below left]{$B$};
  \draw[thick,blue] ($2*(i)+(j)$) node{$\times$} node[below left]{$C$};
  \draw[thick,blue] ($2*(j)$) node{$\times$} node[below left]{$D$};
\end{tikzpicture}

\end{document} 
