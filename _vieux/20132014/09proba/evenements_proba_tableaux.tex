\documentclass[12pt]{article}

\usepackage{pablo}
\usepackage{tabularx}
\usepackage{pifont}
\usepackage{amssymb}
\usepackage{multicol}

\usepackage[landscape,a5paper,margin=1cm]{geometry}

\pagestyle{empty}

\begin{document}

\begin{activite}~
  \begin{enumerate}
    \item Une urne contient des boules numérotées, blanches \textcircled{1} \textcircled{2} \textcircled{2} et rouges \textcircled{1} \textcircled{2}. On prend une boule au hasard dans l'urne.
      \begin{enumerate}
        \item Quel est la probabilité de l'évènement \evenement{A}{Tirer une boule numérotée 0}.
        \item Quel est la probabilité de l'évènement \evenement{B}{Tirer une boule blanche ou une boule rouge}.
        \item \evenement{C_1}{Tirer une boule  blanche} et \evenement{C_2}{Tirer une boule rouge} ;

          \evenement{D_1}{Tirer une boule numérotée 1} et \evenement{D_2}{Tirer une boule numérotée 2}.
        \item \begin{multicols}{2}
          \evenement{E_1}{Tirer une boule blanche},\\
          \evenement{E_2}{Tirer une boule rouge},\\
          \evenement{E_3}{Tirer une boule blanche et rouge},\\
          \evenement{E_4}{Tirer une boule blanche ou rouge}.

          \evenement{F_1}{Tirer une boule rouge},\\
          \evenement{F_2}{Tirer une boule numérotée 1},\\
          \evenement{F_3}{Tirer une boule rouge numérotée 1},\\
          \evenement{F_4}{Tirer une boule rouge ou numérotée 1}.
        \end{multicols}
      \end{enumerate}

    \item Soit une expérience aléatoire d'univers $\Omega$, et des évènements $A$ et $B$ quelconques. En vous aidant des réponses à la question précédente, conjecturer :
      \begin{inparaenum}
        \item la valeur $P(\emptyset)$ ;
        \item la valeur $P(\Omega)$ ;
        \item une relation entre $P(\bar A)$ et $P(A)$ ;
        \item une relation entre $P(A)$, $P(B)$, $P(A\cap B)$ et $P(A\cup B)$.
      \end{inparaenum}
\end{enumerate}
\end{activite}

\vspace{\stretch{1}}
\dotfill
\raisebox{-0.22\baselineskip}{\ding{34}}
\dotfill
\dotfill
\raisebox{-0.22\baselineskip}{\ding{34}}
\dotfill
\dotfill
\raisebox{-0.22\baselineskip}{\ding{34}}
\dotfill
\vspace{\stretch{1}}

\begin{activite}Dans une classe de secondes de 29 élèves, 16 élèves pratiquent le ski, 11 élèves le surf, et 4 élèves les deux sports.
  \begin{enumerate}
    \item Combien d'élèves ne pratiquent ni l'un ni l'autre ? Quelle est la probabilité qu'un élève pris au hasard ne pratique aucun des sports ?
    \item Quelle est la probabilité qu'un élève pratiquant au hasard parmi ceux pratiquant le ski pratique aussi le surf ?
  \end{enumerate}
\end{activite}

\end{document}
