\documentclass[12pt]{article}

\usepackage{pablo}
\usepackage[a5paper,landscape,margin=1cm]{geometry}

\pagestyle{empty}

\begin{document}

\begin{activite}
Une trousse contient cinq stylos, indiscernables au toucher, verts ou bleus. On peut, autant de fois qu'on le désire, prendre au hasard un stylo dans la trousse, le regarder, et le remettre dedans.

Comment faire pour deviner le nombre de stylos verts et le nombre de stylos bleus présents dans la trousse ?
\end{activite}

\vfill

\section{Vocabulaire}
\begin{definition}~
  \begin{itemize}
    \item Une \blanc{expérience aléatoire} est une expérience faisant intervenir le hasard, et comportant plusieurs issues (pouvant donner plusieurs résultat). On ne connaît pas, à priori, le résultat d'une telle expérience.
    \item Une \blanc{issue} est un résultat possible de l'expérience aléatoire.
    \item L'\blanc{univers} est l'ensemble de toutes les issues. Il est généralement noté $\Omega$.
    \item Un \blanc{évènement} est un ensemble d'issues.
    \item Un \blanc{évènement élémentaire} est un évènement ne comportant qu'une seule issue.
  \end{itemize}
\end{definition}
\end{document}
