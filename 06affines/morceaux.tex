\documentclass[14pt, aspectratio=34]{beamer}

\usepackage[utf8x]{inputenc}
\usepackage[T1]{fontenc}
\usepackage[francais]{babel}

\usepackage{tabularx}

\useoutertheme{infolines}
\usecolortheme{crane}

\institute{Lycée Marie Curie}
\date{2015}
\logo{}
\title{Fonctions affines}

\begin{document}


\section{Fonction affine par morceaux}
\begin{frame}
  Soit $f$ la fonction définie sur $\mathbb{R}$ par :
  \[
    f:x\mapsto\left\{
    \begin{array}{cl}
    -x-2 & \text{ si $x\leq -1$} \\
    \frac{x}{2}-\frac{1}{2} & \text{ si $x>-1$}
    \end{array}
    \right.
  \]
  \begin{enumerate}
    \item Dresser le tableau de signes de $f$ sur $\left] -\infty;-1 \right]$.
    \item Dresser le tableau de signes de $f$ sur $\left] 1;+\infty \right[$.
      \item En déduire les solutions de $f(x)\leq0$ sur $\mathbb{R}$.
  \end{enumerate}
\end{frame}

\end{document}
