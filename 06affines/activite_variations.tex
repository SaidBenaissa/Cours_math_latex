\documentclass[14pt, aspectratio=34]{beamer}

\usepackage[utf8x]{inputenc}
\usepackage[T1]{fontenc}
\usepackage[francais]{babel}

\usepackage{tabularx}

\useoutertheme{infolines}
\usecolortheme{crane}

\institute{Lycée Marie Curie}
\date{2015}
\logo{}
\title{Fonctions affines}

\begin{document}


\section{Variations}
\begin{frame}
  \begin{enumerate}
    \item Sur un même repère orthonormé (les abscisses allant de -4 à 4, et les ordonnées aussi), tracer les courbes représentatives
      des fonctions suivantes :
  \begin{columns}
    \begin{column}{.4\textwidth}
      \begin{itemize}
        \item $f:x\mapsto 2x+1$
        \item $g:x\mapsto -x$
        \item $h:x\mapsto \frac{1}{2}x$
      \end{itemize}
    \end{column}
    \begin{column}{.4\textwidth}
      \begin{itemize}
        \item $p:x\mapsto 0,2x-2$
        \item $q:x\mapsto -0,5x-1$
        \item $r:x\mapsto 4$
      \end{itemize}
    \end{column}
    \end{columns}

    \item Par lecture graphique, déterminer les variations de chacune d'entre
      elles.
    \item Conjecturer : Comment déterminer les variations d'une fonction
      affine à partir de son expression ?
    \item Application : Sans tracer la fonction ni faire de calculs, deviner
      les variations de la fonction $t:x\mapsto -7x+4$.
  \end{enumerate}
\end{frame}

\section{Signe}
\begin{frame}
  Soit une fonction affine $f:x\mapsto ax+b$.
  \begin{enumerate}
    \item Résoudre $f(x)=0$.
    \item Supposons que $a>0$.
      \begin{enumerate}[a.]
        \item Tracer l'allure de la courbe de $f$.
        \item Quel est le signe de $f$ si $x>-\frac{b}{a}$ ?
        \item Quel est le signe de $f$ si $x<-\frac{b}{a}$ ?
      \end{enumerate}
    \item Même question pour $a<0$.
  \end{enumerate}
  
\end{frame}

\end{document}
