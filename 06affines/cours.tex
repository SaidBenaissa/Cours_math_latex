\section{Définitions}

\begin{definition}
  Une \emph{fonction affine} est une fonction de la forme $x\mapsto ax+b$, où $a$ et $b$ sont réels. Elle est définie sur $\mathbb R$.

  Quand $b=0$, la fonction est de la forme $x\mapsto ax$, et on dit alors que la fonction est \emph{linéaire}.
\end{definition}

\begin{definition}Soit une fonction affine $f:x\mapsto ax+b$, et $\cal D$ sa courbe représentative.
  \begin{itemize}
    \item Le réel $a$ est appelé \emph{coefficient directeur}.
    \item Le réel $b$ est appelé \emph{ordonné à l'origine}.
  \end{itemize}
\end{definition}


\section{Variations}

\begin{propriete}
  Soit $f:x\mapsto ax+b$ une fonction affine.
  \begin{itemize}
    \item si $a>0$, la fonction est croissante sur $\mathbb R$;
    \item si $a=0$, la fonction est constante sur $\mathbb R$;
    \item si $a<0$, la fonction est décroissante sur $\mathbb R$.
  \end{itemize}
\end{propriete}

TODO signe

\section{Fonction définie par deux points} TODO à clarifier

\begin{propriete}
  Soient $f:x\mapsto ax+b$ une fonction affine, $\cal D$ sa courbe
  représentative, et $A(x_A,y_A)$ et $B(x_B,y_B)$ deux points de $D$. Alors
  $a=\frac{y_B-y_A}{x_B-x_A}$.
\end{propriete}

\begin{methode}[Détermination de l'équation d'une fonction affine] Soit $f$ une fonction affine dont on connaît la représentation graphique $\cal D$.
  Pour calculer $a$, on choisit deux points arbitraires $A$ et $B$ de $\cal
  D$, et on calcule $a=\frac{y_B-y_A}{x_B-x_A}$.

  Pour calculer $b$, TODO.
\end{methode}

\section{Signe}

TODO Signe d'une fonction affine ; Tableau de signes
