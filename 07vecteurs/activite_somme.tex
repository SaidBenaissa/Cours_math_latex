\documentclass[11pt]{article}

\usepackage{pablo}
\usepackage[a6paper,landscape, margin=1cm]{geometry}
\pagestyle{empty}

\begin{document}

\setcounter{section}{1}
\section{Somme de deux vecteurs}

\begin{activite}~

  \begin{center}
\tikzstyle{point}=[draw, shape=circle, fill=black, inner sep=0pt, minimum size=5pt]
\begin{tikzpicture}[scale=0.8,very thick]
  \draw[dotted,color=gray,step=1] (0,0) grid (13,5);
  \draw (1,1) -- (3,3) -- (6,3) -- (4,1) -- (1, 1);
  \node[left] at (1,1) {$A$};
  \node[above left] at (3,3) {$B$};
  \node[above] at (6,3) {$C$};
  \node[below] at (4,1) {$D$};
  \node[below left] at (7,2) {$E$};
  \node[point] at (7,2) {};
\end{tikzpicture}
\end{center}

\begin{enumerate}[(a)]
  \item Placer le point $F$ tel que $\vecteur{EF}=\vecteur{AB}$.
  \item Placer le point $G$ tel que $\vecteur{FG}=\vecteur{BC}$.
  \item Placer le point $H$ tel que $\vecteur{EH}=\vecteur{AC}$.
  \item Que remarquez-vous ? Comment traduire cela par une égalité de vecteurs ?
\end{enumerate}
\end{activite}

\end{document}
