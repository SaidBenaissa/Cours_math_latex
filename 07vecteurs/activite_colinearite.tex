\documentclass[14pt, aspectratio=43]{beamer}

\usepackage{pablo-beamer}

\institute{Lycée Marie Curie}
\date{2015}
\logo{}
\title{Colinéarité}

\begin{document}


\begin{frame}
  \begin{enumerate}
    \item 
      \begin{enumerate}
        \item Placer trois points distincts $A$, $B$, $C$, quelconques.
        \item Placer $D$ tel que $\vecteur{CD}=2\vecteur{AB}$.
        \item Que peut-on dire des droites $\left( AB \right)$ et $\left( CD \right)$ ?
      \end{enumerate}
    \item Réciproque
      \begin{enumerate}
        \item Soient $A$, $B$, $C$, $D$ quatre points tels que $\left( AB \right)$ et $\left( CD \right)$ soient parallèles.
        \item Existe-t-il un nombre $k$ tel que $\vecteur{AB}=k\vecteur{CD}$ ?
      \end{enumerate}
    \item Conjecturer une propriété : «~Deux droites $\left( AB \right)$ et $\left( CD \right)$ sont parallèles si et seulement si \ldots~»
  \end{enumerate}
\end{frame}

\begin{frame}
  \begin{enumerate}
    \item 
      \begin{enumerate}
        \item Placer deux points distincts $A$ et $B$.
        \item Placer $C$ tel que $\vecteur{AC}=3\vecteur{AB}$.
        \item Que peut-on dire des points $A$, $B$ et $C$. Justifier.
      \end{enumerate}
    \item Réciproque
      \begin{enumerate}
        \item Soient $A$, $B$ et $C$ trois points alignés.
        \item Existe-t-il un nombre $k$ tel que $\vecteur{AB}=k\vecteur{AC}$ ?
      \end{enumerate}
    \item Conjecturer une propriété : «~Trois points $A$, $B$ et $C$ sont alignés si et seulement si \ldots~»
  \end{enumerate}
\end{frame}

\end{document}
