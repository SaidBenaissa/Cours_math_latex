\documentclass[14pt]{beamer}

\usepackage{pablo-beamer}

\institute{Lycée Marie Curie}
\date{2015}
\logo{}
\title{Propriétés des translations}

\begin{document}

\begin{frame}
  \begin{enumerate}
    \item Construire $M'$, $N'$, $O'$, images respectives de $M$, $N$, $O$ par la translation qui transforme $A$ en $B$.
    \item Construire $P$, image de $O$ par la translation qui transforme $M$ en $M'$. Que constate-t-on ?
    \item 
      \begin{enumerate}
        \item Que peut on-dire des points $O$, $M$ et $N$ ? Que peut-on dire des points $O'$, $M'$, $N'$ ?
        \item En déduire une conjecture.
      \end{enumerate}
    \item 
      \begin{enumerate}
        \item Choisir un point quelconque sur le segment $[MN]$, et tracer son image par la translation qui transforme $A$ en $B$. Que constate-t-on ?
        \item En déduire l'image du segment $[MN]$ et de la droite $(MN)$ par la translation qui transforme $A$ en $B$.
      \end{enumerate}
    \item 
      \begin{enumerate}
        \item Placer $I$, milieu du segment $[MN]$.
        \item Construire $I'$, image de $I$ par la translation qui transforme $A$ en $B$.
        \item Que constate-t-on ? En déduire une conjecture.
      \end{enumerate}
  \end{enumerate}
\end{frame}

\end{document}
