\documentclass[14pt, aspectratio=43]{beamer}

\usepackage{pablo-beamer}

\institute{Lycée Marie Curie}
\date{2015}
\logo{}
\title{Milieu et Relation de Chasles}

\begin{document}


\begin{frame}
  \begin{columns}
    \begin{column}{.5\textwidth}
      Soient $A$, $B$ et $C$ trois points distincts, et $I$ et $J$ les milieux respectifs de $\left[ AB \right]$ et $\left[ BC \right]$. Le but de l'exercice est de trouver une relation entre $\vecteur{IJ}$ et $\vecteur{AC}$.
    \end{column}
    \begin{column}{.5\textwidth}
      \begin{center}
        \begin{tikzpicture}[very thick, scale=1.2]
          \coordinate (A) at (0,0);
          \coordinate (B) at (3,1);
          \coordinate (C) at (4,-1);
          \coordinate (I) at ($.5*(A)+.5*(B)$);
          \coordinate (J) at ($.5*(C)+.5*(B)$);

          \draw (A) node[left]{$A$}
          -- (B) node[above]{$B$}
          -- (C) node[right]{$C$};
          \draw[dashed,-latex] (I) node[above left]{$I$}
          -- (J) node[above right]{$J$};
          \draw[dashed, -latex] (A) -- (C);
        \end{tikzpicture}
      \end{center}
    \end{column}
  \end{columns}

  \begin{enumerate}
    \item Avec la relation de Chasles, décomposer $\vecteur{IJ}$ avec le vecteur $B$.
    \item Quelle est la relation entre $\vecteur{AI}$ et $\vecteur{IB}$ ? En déduire la relation entre $\vecteur{AB}$ et $\vecteur{IB}$.
    \item De même, montrer que $\vecteur{BJ}=\frac{1}{2}\vecteur{BC}$.
    \item Bilan : Montrer que $\vecteur{IJ}=\frac{1}{2}\vecteur{AC}$.
  \end{enumerate}

\end{frame}
\end{document}
