\documentclass[12pt]{article}

\usepackage{pablo}
\usepackage[a6paper,landscape,margin=1.0cm]{geometry}

\pagestyle{empty}
\begin{document}

%\begin{definition}
%  L'\emph{intervalle de fluctuation} au seuil de 95\%, relatif aux
%  échantillons de taille $n$, est l’intervalle centré autour de $p$,
%  proportion du caractère dans la population, où se situe, avec une
%  probabilité égale à $0,95$, la fréquence observée.
%\end{definition}

\begin{defprop}
  Pour un échantillon de taille $n\geq25$, et une proportion $p$ du
  caractère appartenant à $[0,2;0,8]$, la fréquence observée d'apparition d'un caractère dans l'échantillon appartient à l'\blanc{intervalle de fluctuation} \blanc{$\left[p-\frac{1}{\sqrt{n}};p+\frac{1}{\sqrt{n}}\right]$} avec une probabilité d'au moins \blanc{$0,95$}.
\end{defprop}

\begin{methode}[Prise de décision]
  Étant donné un échantillon de taille $n\geq25$, on se demande s'il est compatible avec un modèle donné, où le caractère apparait avec une probabilité $p\in\left[ 0,2;0,8 \right]$. On note $f$ la fréquence d'apparition du caractère dans cet échantillon.
  \begin{itemize}
    \item Si $f\notin\left[ p-\frac{1}{\sqrt{n}};p+\frac{1}{\sqrt{n}} \right]$, \phantom{on peut rejeter l'hypothèse que l'échantillon soit compatible avec le modèle.}\\[1cm]
    \item Si $f\in\left[ p-\frac{1}{\sqrt{n}};p+\frac{1}{\sqrt{n}} \right]$, \phantom{on ne peut pas rejeter l'hypothèse que l'échantillon soit compatible avec le modèle.}
  \end{itemize}
\end{methode}

\end{document}
