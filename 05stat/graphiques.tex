\documentclass[12pt]{article}

\usepackage{pablo}
\usepackage{pgfplots}
\usepackage{numprint}
\usepackage{calc}
\usepackage[paperwidth=210mm,paperheight={297mm/4},margin=1cm]{geometry}
\usetikzlibrary{calc}

\pagestyle{empty}
\begin{document}

\setcounter{section}{1}
\section{Représentations graphiques}

Dans cette partie, nous travaillerons sur les exemples suivants.

\begin{enumerate}[(a)]
  \item Répartition des résidences principales selon le nombre d'habitants (source : \textsc{Insee}).

    \begin{center}
      \begin{tabular}{c||c|c|c|c|c|c}
        Nombre d'habitants &
        1 & 2 & 3 & 4 & 5 & 6 et plus
        \\
        \hline
        Fréquence (en \%) &
        22,1 &
        27,6 &
        19,1 &
        15,3 &
        8,2 &
        7,6
        \\
      \end{tabular}
    \end{center}

  \item Répartition de la population française par âge, en 2013 (source : \textsc{Insee}).

    \begin{center}
      \begin{tabular}{c||c|c|c|c|c|c}
        Âges & $\left[ 0;20 \right[$
          & $\left[ 20;40 \right[$
            & $\left[ 40;60 \right[$
              & $\left[ 60;80 \right[$
                & $\left[ 80;100 \right[$
                  & $\left[ 100;120 \right[$ \\
                    \hline
                    Effectifs & \numprint{15 533 825}
                    & \numprint{15 699 248}
                    & \numprint{17 114 131}
                    & \numprint{11 671 532}
                    & \numprint{3 621 065}
                    & \numprint{19 807}
                    \\
                  \end{tabular}
                \end{center}
\end{enumerate}

\end{document}
