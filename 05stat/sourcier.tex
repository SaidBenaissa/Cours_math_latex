\documentclass[14pt, aspectratio=43]{beamer}

\usepackage[utf8x]{inputenc}
\usepackage[T1]{fontenc}
\usepackage[francais]{babel}
\usepackage{tikz}

\useoutertheme{infolines}
\usecolortheme{crane}

\newcommand{\touche}[1]{\framebox{#1}}

\institute{Lycée Marie Curie}
\date{2014}
\logo{}
\title{Échantillonnage}

\begin{document}


\begin{frame}
\end{frame}

\begin{frame}
  Un «~sourcier~» accepte de se préter à une expérience, et essaye de déterminer si un seau recouvert contient ou non de l'eau. Sur 50 expériences, il a trouvé la réponse correcte 31 fois.

  Le sourcier a-t-il un don ?
\end{frame}

\begin{frame}
  Par groupes de \emph{deux} élèves, simuler 50 expériences aléatoires.

  Pour cela, on répète 50 fois l'expérience suivante :

  \begin{enumerate}
    \item Tirer un nombre entre 0 et 1 au hasard à la calculatrice : \touche{Math} \touche{PRB} \touche{rand}.
    \item Si ce nombre est supérieur à 0,5, compter un succès ; sinon, compter un échec.
  \end{enumerate}
\end{frame}

\end{document}
