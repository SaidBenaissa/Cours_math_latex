\section{Vocabulaire}

TODO

\begin{itemize}
  \item Population, individu
  \item Caractère, qualitatif ou quantitatif, discret ou continu
  \item classe, intervalle, étendue
  \item Effectif total, d'une valeur
  \item Fréquence, effectifs cumulés croissants,fréquences cumulées croissantes
\end{itemize}

\section{Représentations graphiques}

\subsection{Histogramme}

TODO Description, exemple.

\subsection{Nuage de points}

TODO Description, exemple.

\subsection{Courbe des effectifs cumulés}

TODO Description, exemple.

\subsection{Diagramme circulaire}

TODO Description, exemple.


\section{Indicateurs d'une série statistique}

\subsection{Définitions}

\begin{definition}[Moyenne]
  La \emph{moyenne} d'une série de caractères $x_1, x_2, \ldots x_k$ et d'effectifs $n_1, n_2 \ldots n_k$ est la valeur notée $\bar x$ valant :
  \[{\bar x}=\dfrac{n_1x_1+n_2x_2+\ldots+n_kx_k}{n_1+n_2+\ldots+n_k}.\]
\end{definition}

\begin{definition}[Médiane]
  La \emph{médiane} d'une série est une valeur $m$ telle que la moitié des effectifs étudiés ait une valeur inférieure ou égale à $m$, et l'autre moité ait une valeur supérieure ou égale.
\end{definition}

\begin{remarque}
  La moyenne et la médiane sont des indicateurs de \emph{position}.
\end{remarque}

\begin{definition}[Quartiles]~
  \begin{itemize}
    \item Le \emph{premier quartile} d'une série est la plus petite valeur telle qu'au moins $25\%$ des valeurs lui soit inférieure ou égale.
    \item Le \emph{troisième quartile} d'une série est la plus petite valeur telle qu'au moins $75\%$ des valeurs lui soit inférieure ou égale.
  \end{itemize}
\end{definition}

\begin{remarque}
  Les quartiles sont des indicateurs de \emph{dispersion}
\end{remarque}

\subsection{Méthodes}
\subsubsection{Calcul de la médiane}
On commence par trier les caractères par ordre croissant.
\begin{itemize}
  \item Si la série étudiée a un effectif $N$ impair, la médiane est la valeur du caractère numéroté $\frac{N+1}{2}$.
  \item Si la série étudiée a un effectif $N$ pair, la médiane est la moyenne des caractères numérotés $\frac{N}{2}$ et $\frac{N}{2}+1$.
\end{itemize}

\subsubsection{Calcul des quartiles}

On trie les effectifs par ordre croissant, et on applique la définition.

\begin{exemple}
Calculer la médiane et les quartiles des salaires des entreprises A et B.

  A :
  \begin{tabular}{r|c|c|c|c|c|c|c}
    Salaire  & 1450 & 1500 & 1550 & 1600 & 1650 & 1700 & 1750 \\
    \hline
    Effectif & 5    &  2   &  5   &  7   &  3   &  6   & 7    \\
  \end{tabular}

  B :
  \begin{tabular}{r|c|c|c|c|c|c|c}
    Salaire  & 1450 & 1460 & 1470 & 1580 & 1600 & 2000 & 5000 \\
    \hline
    Effectif & 8    &  2   & 15   &  2   &  5   &  2   & 1    \\
  \end{tabular}
\end{exemple}

\subsection{Quel indicateur choisir ?}

TODO

\begin{itemize}
  \item Sensibilité aux extrèmes
  \item Longue traîne
\end{itemize}

\section{Échantillon}

\begin{activite}
  Je lance 30 fois une pièce de monnaie, et j'obtiens 8 fois Pile et 22 fois Face. Je me demande si cette pièce est équilibrée ou non.

  Proposer une réponse, ou une méthode pour répondre à cette question.
\end{activite}


\begin{definition}
  On appelle \emph{échantillon de taille $n$} les résultats de $n$ répétitions indépendantes d'une même expérience aléatoire.
\end{definition}

\begin{exemple}~
  \begin{itemize}
    \item Pour évaluer la qualité des produits à la sortie d'une usine, on en prélève 100 au hasard, et pour chacun d'entre eux, on détermine s'il est considéré défectueux ou non.
    \item Pour tenter de deviner les résultats d'une élection, on prélève au hasard 1000 personnes dans la population, et on leur demande quelle est leur intention de vote.
  \end{itemize}
\end{exemple}

\section{Intervalle de fluctuation}

\begin{activite}
  J'affirme que 95 fois sur 100, si on lance 30 fois de suite une pièce de monnaie équilibrée, on obtient entre 10 et 20 fois Pile (inclus).

  Étant donné cette nouvelle information, que peut-on dire d'une pièce qui a donné 8 pile en 30 lancers ?
\end{activite}

\begin{definition}
  L'\emph{intervalle de fluctuation} au seuil de 95\%, relatif aux
  échantillons de taille $n$, est l’intervalle centré autour de $p$,
  proportion du caractère dans la population, où se situe, avec une
  probabilité égale à $0,95$, la fréquence observée.
\end{definition}

\begin{propriete}
  Pour un échantillon de taille $n\geq25$, et une proportion $p$ du
  caractère appartenant à $[0,2;0,8]$, la fréquence observée d'apparition du caractère dans l'échantillon appartient à l'intervalle $\left[p-\frac{1}{\sqrt{n}};p+\frac{1}{\sqrt{n}}\right]$ avec une probabilité d'au moins $0,95$.
\end{propriete}

\begin{exemple}[Estimation d'une proportion inconnue]
  TODO
\end{exemple}

\begin{exemple}[Prise de décision]
  TODO
\end{exemple}
