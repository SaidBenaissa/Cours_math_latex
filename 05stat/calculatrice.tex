\documentclass[11pt]{article}

\usepackage{pablo}
\usepackage{multicol}
\usepackage[a5paper,margin=1.5cm]{geometry}

\newcommand{\cell}[1]{\texttt{#1}}
\newcommand{\touche}[1]{\framebox{#1}}

\pagestyle{empty}
\begin{document}

\begin{center}
  \large
  Utilisation de la calculatrice

  ~
  \hrule
\end{center}

\begin{exercice}[Utilisation de la calculatrice]
  Le but de l'exercice est de calculer, à la calculatrice, les indicateurs (médiane, quartiles, écart interquartile, moyenne, valeurs extrêmes, etc.) des deux séries suivantes.
  \begin{enumerate}[(a)]
    \item 13, 7, 24, 20, 12, 14, 15, 15, 10, 22, 5, 6
    \item \begin{tabular}{l|c|c|c|c|c|c|c|c|c}
        Valeurs & 5 & 7 & 13 & 14 & 15 & 18 & 19 & 22 & 23 \\
        \hline
        Effectifs & 1 & 2 & 2 & 6 & 7 & 10 & 32 & 2 & 3 \\
      \end{tabular}
  \end{enumerate}
    
  \begin{enumerate}[{Série} (a)]
    \item \begin{enumerate}[1.]
        \item Entrer la liste des valeurs : \touche{Stat} \touche{Edite}, puis rentrer les valeurs dans la colonne $L_1$.
        \item Retourner dans la fenêtre principale : \touche{2nde} \touche{Quitter}.
        \item Utiliser la fonction \texttt{Stats 1-Var} sur la liste : \touche{Stats} \touche{CALC} \touche{Stats 1-Var}, puis mettre la liste en argument : \touche{2nde} \touche{$L_1$}.
        \item Déterminer les : moyenne, minimum et maximum, premier et troisième quartile, médiane.
      \end{enumerate}
    \item \begin{enumerate}[1.]
        \item Entrer la liste des valeurs et celle des effectifs : \touche{Stat} \touche{Edite}, puis rentrer les valeurs dans les colonnes $L_1$ et $L_2$.
        \item Utiliser la même fonction \texttt{Stats 1-Var} avec les arguments \touche{$L_1$} et \touche{$L_2$} : suivant le modèle, il faudra soit les mettre sur deux lignes différentes, soit les séparer par une virguale \touche{,}.
        \item Déterminer les mêmes indicateurs qu'à la question précédente.
      \end{enumerate}
  \end{enumerate}
\end{exercice}

\begin{exercice}[Application]
    Deux sportifs Benjamin et Mathilde s'entraînent pour courir un cent mètre. Voici leurs temps (en secondes) aux entraînements.

    \begin{itemize}
      \item 
        Benjamin
        ; 11,2
        ; 14
        ; 15,5
        ; 12,3
        ; 14
        ; 16,2
        ; 17
        ; 15,4
        ; 12,5
        ; 13,2.
      \item 
        Mathilde
        ; 12,8
        ; 15
        ; 13,3
        ; 13,7
        ; 14,6
        ; 14,4
        ; 14,5
        ; 13,6
        ; 14,1
        ; 13,6
    \end{itemize}

  \begin{enumerate}
    \item Calculer la médiane et l'écart interquartile pour chacune des séries.
    \item Comparer les deux séries.
  \end{enumerate}
\end{exercice}

\end{document}
