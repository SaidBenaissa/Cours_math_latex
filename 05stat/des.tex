\documentclass{article}

\usepackage{pablo-devoir}
\usepackage{pablo-listings}
\usepackage[a5paper]{geometry}

\title{Lancer de dés et Échantillonnage}
\date{8/01/15}
\classe{2\up{de}14}
\dsnum{}

\newcommand\CR{\texttt{(CR)}}

\begin{document}

\maketitle

\begin{em}
  Un compte-rendu est attendu en fin de séance (sous forme papier, ou électronique), ainsi que l'algorithme que vous aurez écrit. Les réponses aux questions précédées d'un \CR{} doivent apparaitre sur le compte rendu.
\end{em}

On considère l'expérience aléatoire suivante : on lance deux dés à six faces, équilibrés, et on regarde si la somme des deux dés (que l'on appellera \emph{score} par la suite) est supérieure ou égale à 8.

\section{Étude théorique}

\begin{enumerate}
  \item \CR{} Quelle est la probabilité $p$ d'obtenir un score supérieur ou égal à 8 ?
  \item \CR{} On répète 40 fois cette expérience. Quelle est l'intervalle de fluctuation à 95\% ?
  \item \CR{} On a lancé 40 fois deux dés qu'on suspecte d'être truqués, et on a obtenu 11 scores supérieurs ou égaux à 8. Peut-on en déduire qu'ils sont truqués ?
\end{enumerate}

\pagebreak

\section{Simulation}

Le but de cette partie est de vérifier que, dans 95\% des cas, la proportion de scores supérieurs ou égaux à 8 est incluse dans l'intervalle de définition.

\begin{enumerate}
  \item Recopier le programme suivant dans l'environnement Python,et exécutez le.

    \CR{} À quoi sert-il ?

    \begin{lstlisting}[language=python,frame=lines,mathescape=true]
    from random import randint
    from math import sqrt

    n = 40

    for i in range(n):
      if randint(1, 6) + randint(1, 6) >= 8:
        print "Au dessus de 8"
      else:
        print "En dessous de 8"
    \end{lstlisting}

  \item Modifier ce programme pour qu'il compte et affiche le nombre de scores au dessus de 8.

    \CR{} Exécuter plusieurs fois ce nouveau programme, et écrire le résultat sur votre compte-rendu.
  \item Modifier ce programme pour qu'il répète 100 fois cette simulation.
  \item Modifier ce programme pour qu'il compte et affiche le nombre de simulations dont la fréquence de score au dessus de 8 soit dans l'intervalle de fluctuation.

    \CR{} Exécuter plusieurs fois ce nouveau programme, et écrire le résultat sur votre compte-rendu.
  \item \CR{} Exécuter plusieurs fois ce programme, et calculer la fréquence obtenue. Est-ce conforme avec les résultats vus en cours ?
\end{enumerate}


\end{document}
