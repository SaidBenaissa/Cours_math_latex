\documentclass[12pt, aspectratio=34]{beamer}

\usepackage[utf8x]{inputenc}
\usepackage[T1]{fontenc}
\usepackage[francais]{babel}

\usepackage{tabularx}

\useoutertheme{infolines}
\usecolortheme{crane}

\institute{Lycée Marie Curie}
\date{2014}
\logo{}
\title{Statistiques descriptives}

\begin{document}


\section{Définitions}
\begin{frame}
  \begin{tabularx}{\textwidth}{||X}
    Au final, près d’un internaute sur deux (47~\%) déclare avoir déjà
    fréquenté au moins une fois des plateformes illégales quelle que soit
    leur nature.
  \end{tabularx}
    \begin{flushright}
      \begin{em}
        \small \emph{La réponse graduée de l’Hadopi a-t-elle un effet sur le
          piratage de musique et de films ? Une étude empirique
        des pratiques de consommation en ligne}, Darmon, Dejean, Pénard, 11/2014
      \end{em}
    \end{flushright}

    \vfill

  On s'intéresse à la statistique « 47~\% » :
  \begin{enumerate}
    \item Quelle est la \emph{population} étudiée ? Donner un exemple d'\emph{individu}.
    \item Quel est le \emph{caractère} étudié ? Est-il \emph{qualitatif} ou \emph{quantitatif} ?
    \item Quelle est la \emph{fréquence} ? Quel est l'\emph{effectif} ?
  \end{enumerate}

\end{frame}

\end{document}
